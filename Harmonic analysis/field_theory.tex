\section{Field theory}
    \subsection{Structure of the category of fields}
        \begin{lemma}[Ring maps from fields are injective] \label{lemma: ring_maps_from_fields_are_injective}
            Every ring homomorphism from a field is injective.
        \end{lemma}
            \begin{proof}
                A field has no non-zero proper ideal, so if $k$ is a field and $R$ is any commutative ring and $\varphi: k \to R$ is a ring homomorphism, then $\ker \varphi = 0$ necessarily. Thus, $\varphi$ must be injective.  
            \end{proof}
        \begin{corollary}[Category of fields]
            There is a non-full subategory $\Fld$ of the category of commutative rings and ring homomorphisms between them, whose objects are fields and morphisms are the (injective) ring maps between them. 
        \end{corollary}

        \begin{definition}[Field extensions] \label{def: field_extensions}
            To give a ring map $\iota: k \hookrightarrow k'$ between fields $k, k'$ is the same as to give a \textbf{field extension}. We will usually be writing $k'/k$, though we do not mean the quotient.
            
            A field extension $k'/k$ is said to be \textbf{finite} if $k'$ is finite-dimensional as a vector space; otherwise, the field extension is said to be \textbf{infinite}. The quantity:
                $$[k' : k] := \dim_k k'$$
            is called the \textbf{degree} of the extension $k'/k$.
        \end{definition}
        We note right away that the notation for degrees of field extensions was not chosen at random. It is meant to highlight the connection to indices of subgroups of Galois groups. More on this later.
        \begin{remark}
            For every base field $k$, the category of field extensions of $k$ is nothing but the coslice ${}^{k/}\Fld$.
        \end{remark}

        Now, let us recall some relevant features of PIDs which shall be of great use to us when we try to construct and analyse examples.
        \begin{definition}[Euclidean valuations and Euclidean domains] \label{def: euclidean_valuations_and_euclidean_domains}
            \cite[Definition V.2.7]{aluffi_chapter_0} A \textbf{Euclidean valuation} on an integral domain $R$ is a function $v: R \to \Z_{\geq 0}$ (we are implicitly equipping $\Z_{\geq 0}$ with the usual ordering) such that:
                \begin{itemize}
                    \item $v(r) = 0$ if and only if $r = 0$,
                    \item for all $a \in R$ and all $b \in R \setminus \{0\}$, there exists $q\in R$ and $r \in R$ such that $a = qb + r$ and such that $v(r) \leq v(b)$ with equality occurring if and only if $r = 0$.
                \end{itemize}
            A pair $(R, v)$ consisting of an integral domain $R$ and a Euclidean valuation $v: R \to \Z_{\geq 0}$ is called a \textbf{Euclidean domain}.
        \end{definition}
        \begin{example}
            Every commutative ring in which the Euclidean algorithm holds (e.g. $R := \Z, k[x]$ for $k$ a field) is automatically a Euclidean domain in the sense of definition \ref{def: euclidean_valuations_and_euclidean_domains}.
        \end{example}
        \begin{lemma}[Eucliean domains are PIDs] \label{lemma: euclidean_domain_are_PIDs}
            \cite[Proposition V.2.8]{aluffi_chapter_0} Every Euclidean domains are PIDs\footnote{A \textbf{principal ideal domain} (PID) is an integral domain wherein all ideals are principal, i.e. generated by a single element.}.
        \end{lemma}
            \begin{proof}
                Let $(R, v)$ be a Euclidean domain. Since the zero ideal $(0)$ is tautologically principal, let us assume that we are working with a non-zero ideal $(0) \not = I \subset R$. Because $v(I) \subset \Z_{\geq 0}$, it makes sense to talk about a non-zero element $b \in I \setminus \{0\}$ such that $v(b) \leq v(x)$ for all $x \in I \setminus \{0\}$; we claim that $I = (b)$ and hence $I$ is principal, as needed. To prove that this is true, note firstly that we automatically have $(b) \subseteq I$ and as such, it shall suffice to only demonstrate that $(b) \supseteq I$. To that end, note that because $R$ is a Euclidean domain, there always exists $q, r \in R$ for every $a \in R$ such that $a = qb + r$ and such that either $v(r) = 0$ or $v(r) < v(b)$ (cf. definition \ref{def: euclidean_valuations_and_euclidean_domains}). But by assumption, $v(b) \leq v(x)$ for all $x \in I \setminus \{0\}$, so $v(b) \leq v(r)$ and as such one has that $v(r) = 0$ and hence $r = 0$ necessarily. Therefore $a = qb$, i.e. $b \mid a$ and so $a \in (b)$ for all $a \in I$, and hence $(b) \supseteq I$ as needed, so we are done.  
            \end{proof}
        \begin{proposition}[Polynomial rings over fields are PIDs] \label{prop: polynomial_rings_over_fields_are_PIDs}
            Let $k$ be any field. Then $k[x]$ shall be a PID.
        \end{proposition}
            \begin{proof}
                We take for granted the fact that $k[x]$ is an integral domain. Because we have the Euclidean Algorithm for univariate polynomials over fields, $k[x]$ has a Euclidean valuation in the sense of definition \ref{def: euclidean_valuations_and_euclidean_domains} and is therefore a Euclidean domain by definition. It is thus a PID by lemma \ref{lemma: euclidean_domain_are_PIDs}. 
            \end{proof}
            
        \begin{definition}[Prime and maximal ideals] \label{def: prime_and_maximal_ideals}
            Let $R$ be a commutative ring. An ideal $\p \subset R$ is \textbf{prime} if and only if $R/\p$ is an integral domain, while an ideal $\m \subset R$ is \textbf{maximal} if and only if $R/\m$ is a field. 
        \end{definition}
        \begin{lemma}[The Third Isomorphism Theorem for rings and ideals] \label{lemma: the_third_isomorphism_theorem_for_rings_and_ideals}
            \cite[Proposition III.3.11]{aluffi_chapter_0} Let $R$ be a ring and $I$ be an arbitrary left/right-$R$-ideal. Then, all left/right-ideals of $R/I$ are of the form $J/I$ for some left/right-$R$-ideal $J$ that contains $I$ properly as a subset. 
        \end{lemma}
            \begin{proof}
                The proof is very similar to that of the Third Isomorphism for groups and/or vector spaces, so we shall be referring the reader to \cite[Proposition III.3.11]{aluffi_chapter_0}.
            \end{proof}
        \begin{remark}[Maximal ideals are prime] \label{remark: maximal_ideals_are_prime}
            Maximal ideals are prime. This can be easily seen via an application of the Third Isomorphism Theorem (cf. lemma \ref{lemma: the_third_isomorphism_theorem_for_rings_and_ideals}) and the fact that fields are certain instances of integral domains by definition. 
        \end{remark}
        \begin{lemma}[Factorisation and prime ideals] \label{lemma: factorisation_and_prime_ideals}
            Let $R$ be a commutative ring. Then an ideal $\p \subset R$ will be prime if and only if for all $a, b \in R$, that $ab \in \p$ implies that either $a \in \p$ or $b \in \p$. 
        \end{lemma}
            \begin{proof}
                An ideal $\p \subset R$ is prime if and only if $R/\p$ is an integral domain (cf. definition \ref{def: prime_and_maximal_ideals}). $R/\p$ is an integral domain if and only if, by definition, for all $\bar{a}, \bar{b} \in R/\p$, that $\bar{a} \bar{b} = 0$ implies that either $\bar{a} = 0$ or $\bar{b} = 0$. This is the case if and only if that $ab \equiv 0 \pmod{p}$ implies that either $a \equiv 0 \pmod{\p}$ or $b \equiv 0 \pmod{\p}$ for all $a, b \in R$. This is equivalent to that that $ab \in \p$ implies that either $a \in \p$ or $b \in \p$ for all $a, b \in R$.
            \end{proof}
        \begin{lemma}[Prime ideals of PIDs] \label{lemma: prime_ideals_of_PIDs}
            If $R$ is a PID then any \textit{non-zero} prime ideal $\p \subset R$ will be of the form $\p = (f)$ for some irreducible element $f \in R$.  
        \end{lemma}
            \begin{proof}
                Suppose for the sake of deriving a contradiction that there exists a prime ideal $\p \subset R$ such that $\p = (f)$ but $f \in R$ is not irreducible. Prime ideals are proper, so $f \not \in R^{\x}$, and as such let us assume that there exists $a, b \in R \setminus R^{\x}$ such that $f = ab$. Because $\p = (f)$ is prime by assumption, that $ab \in \p$ implies that either $a \in \p$ or $b \in \p$, \textit{exclusively}. This implies, respectively, that either $b \nmid f$ or $a \nmid f$. This contradicts the assumption that $f = ab$, so $f$ must be irreducible.
            \end{proof}

        Let $K$ be a field. Since $\Z$ is initial in the category of commutative rings, there is always a unique ring map $f: \Z \to K$. $\im f$ is a subring of a field, and hence an integral domain, and therefore $\ker f \subset \Z$ must be a prime ideal. Since $\Z$ is a PID, we know by lemma \ref{lemma: prime_ideals_of_PIDs} that its prime ideals are either $(0)$ or $(p)$ for some prime number $p$, and in the latter case, the ideal is also maximal. Now, by the First Isomorphism Theorem, we know that:
            $$\im f \cong \Z/\ker f$$
        and so:
            $$
                \im f \cong  
                \begin{cases}
                    \text{$\Z$ if $\ker f = (0)$, i.e. $f$ is injective}
                    \\
                    \text{$\Z/p$ if $\ker f = (p)$ for some prime $p$}
                \end{cases}
            $$
        \begin{definition}[Characteristics of fields] \label{def: characteristics_of_fields}
            In the situation above, we say that the \textbf{characteristic} of $K$, written:
                $$\chara K$$
            is $0$ if $f: \Z \to K$ is injective, and is $p$ if $\im f \cong \Z/p$ for some prime $p$.
        \end{definition}
        \begin{convention}[Finite fields]
            For every prime $p$, since $(p)$ is maximal, $\F_p := \Z/p$ is a field. This is called the \textbf{finite field} with $p$ elements.
        \end{convention}
        \begin{remark} \label{remark: extensions_have_the_same_characteristic_as_base}
            In fact, every field of characteristic $0$ is an extension of $\Q$ as they contain $\Z$ as well as inverses of non-zero elements therein, while every field of characteristic $p$ - for some prime $p$ - must be an extension of $\F_p$.
        \end{remark}
            
        \begin{proposition}[Irreducible elements of PIDs] \label{prop: irreducible_elements_of_PIDs}
            Let $R$ be a PID. Then an element $f \in R$ is irreducible (i.e. it is not a unit and is not the product of two non-invertible elements) if and only if the ideal $(f) \subset R$ is maximal.
        \end{proposition}
            \begin{proof}
                Suppose firstly that $f \in R$ is irreducible and suppose to the contrary - for the sake of deriving a contradiction - that there exists $g \in R$ such that $(g) \supset (f)$ (note the proper containment) and such that the ideal $(g)$ is proper. This implies that $g \mid f$, i.e. there exists some $a \in R$ such that $f = ag$. However, because $f$ has been assumed to be irreducible, either $a$ or $g$ must be a unit of $R$. Respectively, these scenarios imply that $(g) = (f)$ and that $(g) = (1) = R$. Both violate the conditions we imposed earlier upon $g$, so we are done.
                
                Conversely, suppose that $(f) \subset R$ is a maximal ideal. Maximal ideals are prime (cf. remark \ref{remark: maximal_ideals_are_prime}), so the ideal $(f) \subset R$ is prime. $f \in R$ is therefore an irreducible element as a result of $R$ being a PID by assumption (cf. lemma \ref{lemma: prime_ideals_of_PIDs}). 
            \end{proof}
        \begin{corollary}[Maximal ideals of polynomial rings over fields] \label{coro: maximal_ideals_of_polynomial_rings_over_fields}
            Let $k$ be a field. Then, any ideal $\m \subset k[x]$ is maximal if and only if $\m = (f(x))$ for some irreducible $f(x) \in k[x]$.
        \end{corollary}
            \begin{proof}
                Combine propositions \ref{prop: polynomial_rings_over_fields_are_PIDs} and \ref{prop: irreducible_elements_of_PIDs}.
            \end{proof}
        \begin{theorem}[Maximal ideals of polynomial rings over commutative rings] \label{theorem: maximal_ideals_of_polynomial_rings_over_commutative_rings}
            If $R$ is any commutative ring then every maximal ideal of $R[x]$ is of the form $\m + (f(x))$ wherein $\m \subset R$ is any maximal ideal, and $f(x) \in R[x]$ is such that $f(x) \pmod{\m} \in (R/\m)[x]$ is irreducible.
        \end{theorem}
            \begin{proof}
                Let $\kappa_{\m} := R/\m$. Since $\m \subset R$ is maximal, $\kappa_{\m}$ is a field. 
            
                Firstly, consider the canonical surjective ring homomorphism $\pi_{\m}: R[x] \to \kappa_{\m}[x]$ given by $\pi_{\m}(f(x)) := f(x) \pmod{\m}$. Since $\kappa_{\m}$ is a field, $\kappa_{\m}[x]$ is a PID (cf. proposition \ref{prop: polynomial_rings_over_fields_are_PIDs}). Since $\pi_{\m}$ is surjective, one then gets via the Third Isomorphism Theorem (cf. lemma \ref{lemma: the_third_isomorphism_theorem_for_rings_and_ideals}) that every maximal ideal of $R[x]$ are of the form $\pi_{\m}^{-1}(f(x))$ for some irreducible $f(x) \in \kappa_{\m}[x]$. Per the definition of the map $\pi_{\m}$, one has that $\pi_{\m}^{-1}(f(x)) = \m + (f(x))$ wherein it is understood that one has chosen elements $\tilde{a} \in R$ for the coefficients $a$ of $f(x)$ such that $\tilde{a} \equiv a \pmod{\m}$ (i.e. such that $\pi_{\m}(\tilde{a}) = a$). This concludes the proof. 
            \end{proof}
        \begin{remark}
            In the setting of theorem \ref{theorem: maximal_ideals_of_polynomial_rings_over_commutative_rings}, even if $f \pmod{\m} \in (R/\m)[x]$ is \textit{not} irreducible, $f \in R[x]$ may still be irreducible. See example \ref{example: irreducibility_modulo_p}.
        \end{remark}
        
        \begin{proposition}[Generic irreducibility] \label{prop: generic_irreducibility}
            Let $R$ be an integral domain and let $K := \Frac R$. If $f \in R[x]$ is irreducible, then its image under the canonical map $R[x] \to K[x]$ will also be irreducible.
        \end{proposition}
            \begin{proof}
                If $f \in R[x]$ is irreducible then because $R[x]/f \cong R$ and because the RHS is an integral domain by assumption, $(f) \subset R[x]$ must be a prime ideal. Since $R$ is an integral domain, $(0) \subset R$ is a prime ideal, and its field of fractions $K$ is well-defined as the localisation $R_{(0)}$. Localisation commutes with quotients, so $K[x]/f \cong K$, which implies that $(f) \subset K[x]$ is a maximal ideal. By corollary \ref{coro: maximal_ideals_of_polynomial_rings_over_fields}, this implies that $f$ is also irreducible over $K$.
            \end{proof}
        \begin{example}[Irreducibility modulo $p$] \label{example: irreducibility_modulo_p}
            A polynomial $f \in \Z[x]$ is irreducible if there exists a prime $p \in \Spec \Z$ such that the image $\bar{f} \in \F_p[x]$ of $f$ under the quotient map $\Z[x] \to \F_p[x]$ is irreducible. By proposition \ref{prop: generic_irreducibility}, this also means that $f \in \Q[x]$ is irreducible.

            The converse statement may not be true. If $p \not = 2$ is a prime then $x^2 + p \in \Z[x]$ is irreducible, but $x^2 + p \equiv x^2 \pmod{p} \in \F_p[x]$ is reducible whenever $p \not = 2$.
        \end{example}

        \begin{definition}[Eisenstein polynomials] \label{def: eisenstein_polynomials}
            We say that a \textit{monic} polynomial $f(x) := \sum_{0 \leq d \leq n} a_d x^d \in \Q[x]$ is \textbf{Eisenstein} at a prime $p \in \Z$ if and only if $p^2 \nmid a_0^2$ and $p \mid a_d$ for all $1 \leq d \leq n - 1$.
            
            Eisenstein's Irreducibility Criterion thus reads: if $f(x) \in \Q[x]$ is Eisenstein then it will be irreducible (over $\Q$), though the converse may fail.
        \end{definition}

        Now that we have a reminder of how to deal with irreducible polynomials, let us see some examples of field extensions.
        
        \begin{definition}[Algebraic and transcendental extensions] \label{def: algebraic_and_transcendental_extensions}
            A field extension $K/k$ is said to be algebraic if and only if there exists a set $I$ and a (necessarily maximal) ideal $\m \subset k[\{x_i\}_{i \in I}]$ such that:
                $$K \cong k[\{x_i\}_{i \in I}]/\m$$
            Otherwise, $K/k$ will be called a \textbf{trascendental extension}.
        \end{definition}
        \begin{convention}
            If $K/k$ is a field extension and $\{\alpha_i\}_{i \in I} \subset K$ is a subset, then we shall write $k(\{\alpha_i\}_{i \in I})$ for the smallest subfield of $K$ that contains both $k$ and $\{\alpha_i\}_{i \in I}$. $k(\{\alpha_i\}_{i \in I})$ shall be called the field generated by $\{\alpha_i\}_{i \in I}$ over $k$, and it is an extension of $k$. 

            If $|I| = 1$ then we say that the extension is \textbf{simple} if it is non-trivial, i.e. not $k/k$.
        \end{convention}
        \begin{lemma}[Classification of simple extensions] \label{lemma: classification_of_simple_extensions}
            Let $k$ be a field. Every simple extension of $k$ is either of the form $k[x]/f$ for some irreducible $f \in k[x]$, or of the form $k(x)$.
        \end{lemma}
            \begin{proof}
                Let $K/k$ be a field extension, let $\alpha \in K \setminus k$, and consider the simple extension $k(\alpha)/k$. Without any loss of generality, set $K := k(\alpha)$. Next, consider the ring homomorphism:
                    $$\varphi: k[x] \to k(\alpha)$$
                given by $\varphi(t) := \alpha$. Since $k(\alpha)$ is a field, $\im \varphi$ is a subring of a field, hence an integral domain. $\ker \varphi$ is therefore a prime ideal of $k[x]$, and by lemma \ref{lemma: prime_ideals_of_PIDs}, $\ker \varphi$ is either $(0)$ or $(f)$ for some irreducible $f \in k[x]$. 

                If $\ker \varphi = (f)$ for some irreducible $f \in k[x]$, then because $(f) \subset k[x]$ is a maximal ideal by corollary \ref{coro: maximal_ideals_of_polynomial_rings_over_fields}, $\im f \cong k[x]/f$ is a field that contains both $k$ and $\alpha$. However, since $k(\alpha)$ is the smallest such field, we must then have:
                    $$k[x]/f \cong k(\alpha)$$

                If $\ker \varphi = (0)$ then $k(\alpha)$ will be the smallest field that contains $k[x]$ as a subring. By the universal property of localisations, $k(\alpha) \cong k(t)$, since $k(t) \cong k[x]_{(0)}$.
            \end{proof}
        \begin{corollary}[Finite extensions are algebraic] \label{coro: finite_extensions_are_algebraic}
            Let $k$ be a field. Any finite extension $K/k$ is necessarily algebraic.
        \end{corollary}
            \begin{proof}
                Suppose for the sake deriving a contradiction that there is a transcendental yet finite extension $K/k$. However, the smallest transcendental extension of $k$ is $k(x)$ by lemma \ref{lemma: classification_of_simple_extensions}, since a field extension must be generated by at least one element. $k(x)/k$, however, is infinite, so we have a contradiction and are done. 
            \end{proof}

        \begin{lemma}[Towers of algebraic extensions] \label{lemma: towers_of_algebraic_extensions}
            Consider a tower of field extensions as follows:
                $$
                    \begin{tikzcd}
                        {k''} \\
                        {k'} \\
                        k
                        \arrow[from=2-1, to=1-1]
                        \arrow[from=3-1, to=2-1]
                    \end{tikzcd}
                $$
            \begin{enumerate}
                \item If $k'/k$ is an algebraic extension and $k''/k$ is another algebraic extension, then $k''/k$ will also be algebraic.
                \item However, if either extension is transcendental, then $k''/k$ will be transcendental as well.
                \item If $k''/k$ is algebraic and either $k'/k$ or $k''/k'$ is algebraic, then $k''/k$ and $k'/k$ will be algebraic too, respectively. 
            \end{enumerate}
        \end{lemma}
            \begin{proof}
                \begin{enumerate}
                    \item If $k'/k$ is algebraic then there shall exist some set $I$ and a maximal ideal $\m \subset k[\{x_i\}_{i \in I}]$ such that $k' \cong k[\{x_i\}_{i \in I}]/\m$. If $k''/k'$ is also algebraic, then there shall exist some set $J$ and a maximal ideal $\n \subset k'[\{y_j\}_{j \in J}]$ such that $k'' \cong k'[\{y_j\}_{j \in J}]/\n \cong k[\{x_i\}_{i \in I} \cup \{y_j\}_{j \in J}]/(\m + \n)$. Since $k''$ is a field, this shows that $\m + \n \subset k[\{x_i\}_{i \in I} \cup \{y_j\}_{j \in J}]$ is a maximal ideal too (one can also apply theorem \ref{theorem: maximal_ideals_of_polynomial_rings_over_commutative_rings} iteratively when $I, J$ are countable), and hence $k''/k$ is also algebraic. 
                    \item Without any loss of generality, it is enough to consider $k' \cong k(x)$, since every other transcendental extension of $k$ will be of degree larger than that of $k(x)/k$. Then, even if $k''/k'$ is algebraic, there shall still exist an injective ring map $k[x] \hookrightarrow k' := k(x) \hookrightarrow k''$, and hence $k''/k$ is also transcendental by lemma \ref{lemma: classification_of_simple_extensions}. 
                    \item If $k''/k$ is algebraic and $k''/k'$ is algebraic, but $k'/k$ is not algebraic, then by the previous case, $k''/k$ will have to also be transcendental. This implies that there is an injective and surjective ring map, i.e. an isomorphism, $k[x] \xrightarrow[]{\cong} k''$. But the LHS is not a field, so we have a contradiction, and thus $k'/k$ must also be algebraic. 
                \end{enumerate}
            \end{proof}
        \begin{remark}[Towers of finite extensions] \label{remark: towers_of_finite_extensions}
            Consider a tower of field extensions as follows:
                $$
                    \begin{tikzcd}
                        {k''} \\
                        {k'} \\
                        k
                        \arrow[from=2-1, to=1-1]
                        \arrow[from=3-1, to=2-1]
                    \end{tikzcd}
                $$
            \begin{enumerate}
                \item If $k'/k$ is an finite extension and $k''/k$ is another finite extension, then $k''/k$ will also be finite.
                \item If $k''/k$ is finite and either $k''/k'$ or $k'/k$ is finite, then the remaining of the three extensions must also be finite.
            \end{enumerate}
        \end{remark}
        
        For now, we will be mostly interested in algebraic extensions, though transcendental extensions are interesting in their own right. There are, however, some subtleties surrounding such extensions, which is why we are deferring the task of discussing them until later.

        Finally, some actual examples.
        \begin{example}[Quadratic extensions] \label{example: quadratic_extensions}
            If $a$ is any square-free positive integer, then $\Q(\sqrt{a}) := \Q[x]/(x^2 - a)$ shall be a finite extension of $\Q$. In particular, a basis for $\Q(\sqrt{a})$ as a $\Q$-vector space is given by reducing the $\Q$-linear basis $\{x^n\}_{n \geq 0}$ of $\Q[x]$ modulo $x^2 - a$ and then picking a representative from each equivalence class:
                $$\{x^n \pmod{x^2 - a}\}_{n \geq 0} = \{1, \sqrt{a}\}$$
            Note that $a \in \Q$, so it is an element of the subspace $\Q \cdot 1 \subset \Q(\sqrt{a})$, and hence not a basis element anymore if we have already chosen $1$ to be a basis element, and from the equivalence class $x \pmod{x^2 - a}$, we have chosen $\sqrt{a}$, but one can choose $-\sqrt{a}$ as well.

            $\Q(\sqrt{a})/\Q$ is also an instance of a simple extension.
        \end{example}
        \begin{example}[Perfect fields] \label{example: perfect_fields}
            A field of characteristic $p > 0$ is called \textbf{perfect} if it contains all $p^{th}$ power roots, i.e. $k[x]/(x^p - a) \cong k$ for all $a \in k$. The \textbf{perfection} of a field $K $of characteristic $p$ is given by adjoining all $p^{th}$ power roots to $K$. This is generally an infinite algebraic extension.
        \end{example}
        \begin{example}[A transcendental extension] \label{example: transcendental_extension}
            If $k$ is any field, then $k(x)$, the field of univariate rational functions with coefficients in $k$ will be a transcendental extension of $k$. Likewise, the field $k(\!(x)\!)$ of univariate formal Laurent series with coefficients in $k$ is also a transcendental extension of $k$.
        \end{example}

        Now, given a simple algebraic extension, how can we compute its degree, and how do we know which irreducible polynomial it arises from ?
        \begin{proposition}[Minimal polynomials] \label{prop: minimal_polynomials}
            Let $k$ be a field and consider simple algebraic extension $k(\alpha)/k$. Then, there shall exist a unique polynomial $f \in k[x]$, called the \textbf{minimal polynomial} of $\alpha$, that is \underline{monic}, irreducible, unique up to rescalings by elements of $k[x]^{\x}$, and is such that there exists a $k$-algebra isomorphism:
                $$k[x]/f \xrightarrow[]{\cong} k(\alpha)$$
                $$x \mapsto \alpha$$
            Furthermore, we have that:
                $$[k(\alpha) : k] = \deg f$$
        \end{proposition}
            \begin{proof}
                By definition, any simple algebraic extension $k(\alpha)/k$ is of the form $k[x]/\m \to k(\alpha)$ for some maximal $k[x]$-ideal $\m$. By corollary \ref{coro: maximal_ideals_of_polynomial_rings_over_fields}, any such ideal must be of the form $\m = (f)$ for some irreducible $f \in k[x]$ that is uniquely determined up to scalings by elements of $k[x]^{\x}$. If we equip $k[x]$ with the basis $\{x^n\}_{n \geq 0}$ and write $f := \sum_{d = 0}^{\deg f} a_d x^d$, then because $k^{\x} \subset k[x]^{\x}$, we can rescale $f$ so that $a_{\deg f} = 1$ to make $f$ monic. That the isomorphism $k[x]/f \xrightarrow[]{\cong} k(\alpha)$ is given by $x \mapsto \alpha$ is forced by definition \ref{def: algebraic_and_transcendental_extensions}.

                Because $k[x]$ is a Euclidean domain, we can always find $q, r \in k[t]$ with $\deg r < \deg f$ such that:
                    $$x^n = p f + r \equiv r \pmod{f}$$
                for all $n \geq 0$. Because $\deg r < \deg f$, $r \in \bigoplus_{0 \leq d < \deg f} k \cdot x^n \pmod{f}$, and since $k[x] \to k[x]/f$ is surjective, we see that $\{x^n \pmod{f}\}_{0 \leq n < \deg f}$ is a basis for $k[x]/f$. We can now conclude that:
                    $$[k(\alpha) : k] = \deg f$$
            \end{proof}
        \begin{corollary}[Roots of minimal polynomials] \label{coro: roots_of_minimal_polynomials}
            If $k(\alpha)/k$ is a simple algebraic extension with $[k(\alpha) : k] = d$ then $\{\alpha^n\}_{0 \leq n < d}$ will be a basis for the $k$-vector space $k(\alpha)$.

            In practice, this means that if $k(\alpha)/k$ is given, then one can compute:
                $$d := [k(\alpha) : k]$$
            as the largest positive integer such that, for all $d' > d$, $\alpha^{d'} \in k$. 
        \end{corollary}
            \begin{proof}
                Let $f$ be the minimal polynomial of $k(\alpha)/k$. By proposition \ref{prop: minimal_polynomials}, we know that $k[x]/f \xrightarrow[]{\cong} k(\alpha)$ with the isomorphism being given by $x \mapsto \alpha$, and the claim follows suite.
            \end{proof}
        \begin{remark}[Minimal polynomials via evaluations] \label{remark: minimal_polynomials_via_evaluations}
            If $k(\alpha)/k$ is a simple algebraic extension with minimal polynomial $f$, then one can also think of the isomorphism $k[x]/f \xrightarrow[]{\cong} k(\alpha)$ given by $x \mapsto \alpha$ as arising from the evaluation map $\ev_{\alpha}: k[x] \to k(\alpha)$ given by $\ev_{\alpha}(g) := g(\alpha)$ for all $g \in k[x]$. Note that $\ker \ev_{\alpha} \cong (f)$ precisely. Through this, one recovers the more traditional approach to minimal polynomials, i.e. that $f$ is the monic (irreducible) polynomial of smallest degree such that $f(\alpha) = 0$.
        \end{remark}
        
        \begin{remark}[Minimal polynomials in field theory vs. in linear algebra]
            Let $d$ be a positive integer and $k$ be a field.
            
            Recall from linear algebra that the \textbf{minimal polynomial} of an $d \x d$ matrix $A \in \Mat_d(k)$ is the monic polynomial of least degree $\mu_A \in k[x]$ such that $\mu_A(A) = 0$. At the same time, if $k(\alpha)/k$ is a simple algebraic extension with minimal polynomial $f \in k[x]$ (in the sense of proposition \ref{prop: minimal_polynomials}), then one gets a $k$-linear operator:
                $$\alpha \cdot: k(\alpha) \to k(\alpha)$$
            given by $\beta \mapsto \alpha \beta$ for all $\beta \in k(\alpha)$. If $d = [k(\alpha) : k]$ then one can represent $\alpha \cdot$ by a $d \x d$ matrix $[\alpha \cdot] \in \Mat_d(k)$. Via remark \ref{remark: minimal_polynomials_via_evaluations}, one sees thus that:
                $$f = \mu_A$$
            which is to say that the two notions of minimal polynomials agree should we interpret the generator $\alpha$ as a linear operator.

            This is a powerful interpretation, since minimal polynomials of square matrices have some rather useful properties, which we list below in no particular order.
            \begin{itemize}
                \item The Cayley-Hamilton Theorem (cf. \cite[Theorem V.6.11]{aluffi_chapter_0}) asserts that if $k$ is a field and $E$ is a $k$-vector space, and if $A \in \End_k(E)$ and $\chi_A \in k[x]$ is its characteristic polynomial, then:
                    $$\chi_A(A) = 0$$
                Since $\deg \mu_A$ is smallest amongst all polynomials $g \in k[x]$ such that $g(A) = 0$, we have as a corollary to the Cayley Hamilton Theorem that:
                    $$\mu_A \mid \chi_A$$
                \item From the fact that $\mu_A \mid \chi_A$, we infer that the set of roots over $k$ of $\mu_A$ is a subset of the set of eigenvalues of $A$, which by definition is nothing but the set of roots $\chi_A$ (cf. \cite[Lemma V.6.14]{aluffi_chapter_0})\footnote{Geometrically, this is saying that $\Spm k[x]/\mu_A$ is a Zariski-closed subvariety of $\Spm k[x]/\chi_A = \Spec(A/k)$ (where $\Spec(A/k)$ denotes the set of eigenvalues over $k$ of $A \in \End_k(E)$).}.
                
                This means that in practice, if one is given a simple algebraic extension $k(\alpha)/k$ of degree $d$ (see corollary \ref{coro: roots_of_minimal_polynomials} for how to compute its degree) then the following steps can be carried out to find an irreducible polynomial $f \in k[x]$ such that $k[x]/f \cong k(\alpha)$.
                \begin{enumerate}
                    \item Firstly, one can compute the corresponding minimal polynomial by firstly representing $\alpha \cdot \in \End_k(k(\alpha))$ as an element $A \in \Mat_d(k)$; one does this by letting $\alpha \cdot$ act on the elements of the basis $\{\alpha^n\}_{0 \leq n < d}$.
                    \item Secondly, set:
                        $$\chi_A(x) = \sum_{n = 0}^d a_n x^n$$
                    Note that because $\chi_A(x) := \det(x\id_d - A)$, we necessarily have that $a_d \in k^{\x}$.
                    \item Since $\deg \mu_A = d$ by proposition \ref{prop: minimal_polynomials} and since $\mu_A \mid \chi_A$, and since minimal polynomials are monic by definition, we can take:
                        $$\mu_A := \frac{1}{a_d} \chi_A$$
                \end{enumerate}
            \end{itemize}
        \end{remark}
        \begin{example}
            Consider the simple algebraic extension $\Q(2^{\frac13})/\Q$.
            
            By corollary \ref{coro: roots_of_minimal_polynomials}, we know that:
                $$\{1, 2^{\frac13}, 2^{\frac23}\}$$
            is a $\Q$-linear basis for $\Q(2^{\frac13})$, so $[\Q(2^{\frac13}) : \Q] = 3$. To see what the matrix representing the $\Q$-linear operator $2^{\frac13} \cdot \in \Mat_3(\Q)$ is, let us see how it acts on the basis vectors $1, 2^{\frac13}, 2^{\frac23}$:
                $$2^{\frac13} \cdot 1 = 2^{\frac13}, 2^{\frac13} \cdot 2^{\frac13} = 2^{\frac23}, 2^{\frac13} \cdot 2^{\frac23} = 2 \cdot 1$$
            If we identify:
                $$\Q(2^{\frac13}) \cong \Q \cdot \begin{pmatrix} 1 \\ 0 \\ 0 \end{pmatrix} \oplus \Q \cdot \begin{pmatrix} 0 \\ 1 \\ 0 \end{pmatrix} \oplus \Q \cdot \begin{pmatrix} 0 \\ 0 \\ 1 \end{pmatrix}$$
            via $1 \mapsto \begin{pmatrix} 1 \\ 0 \\ 0 \end{pmatrix}, 2^{\frac13} \mapsto \begin{pmatrix} 0 \\ 1 \\ 0 \end{pmatrix}, 2^{\frac23} \mapsto \begin{pmatrix} 0 \\ 0 \\ 1 \end{pmatrix}$, then we see that:
                $$
                    [2^{\frac13} \cdot]
                    \begin{pmatrix}
                        1 & 0 & 0
                        \\
                        0 & 1 & 0
                        \\
                        0 & 0 & 1
                    \end{pmatrix}
                    =
                    \begin{pmatrix}
                        0 & 0 & 2
                        \\
                        1 & 0 & 0
                        \\
                        0 & 1 & 0
                    \end{pmatrix}
                $$
            and hence:
                $$
                    [2^{\frac13} \cdot] =
                    \begin{pmatrix}
                        0 & 0 & 2
                        \\
                        1 & 0 & 0
                        \\
                        0 & 1 & 0
                    \end{pmatrix}
                $$
            with respect to the $\Q$-linear basis $\{1, 2^{\frac13}, 2^{\frac23}\} \subset \Q(2^{\frac13})$.

            One then sees easily that:
                $$\chi_{[2^{\frac13} \cdot]}(x) = x^3 - 2$$
            This is already monic and irreducible, and since $\deg \chi_{[2^{\frac13} \cdot]} = [\Q(2^{\frac13}) : \Q] = 3 = \deg \mu_{[2^{\frac13} \cdot]}$, so we can take:
                $$\mu_{[2^{\frac13} \cdot]}(x) = x^3 - 2$$
        \end{example}
        \begin{example}
            Consider the simple algebraic extension $\Q(2^{\frac14})/\Q$.

            By corollary \ref{coro: roots_of_minimal_polynomials}, we see that a basis for the $\Q$-vector space $\Q(2^{\frac14})$ is given by:
                $$\{1, 2^{\frac14}, 2^{\frac12}, 2^{\frac34}\}$$
            so $[\Q(2^{\frac14}) : \Q] = 4$. The $\Q$-linear operator $2^{\frac14} \cdot \in \Mat_4(\Q)$ acts on the basis vectors as:
                $$2^{\frac14} \cdot 1 = 2^{\frac14}, 2^{\frac14} \cdot 2^{\frac14} = 2^{\frac12}, 2^{\frac14} \cdot 2^{\frac12} = 2^{\frac34}, 2^{\frac14} \cdot 2^{\frac34} = 2 \cdot 1$$
            from which we see that:
                $$
                    [2^{\frac14} \cdot] =
                    \begin{pmatrix}
                        0 & 0 & 0 & 2
                        \\
                        1 & 0 & 0 & 0
                        \\
                        0 & 1 & 0 & 0
                        \\
                        0 & 0 & 1 & 0
                    \end{pmatrix}
                $$
            Frmo this, one sees that $\chi_{[2^{\frac14} \cdot]}(x) = x^4 - 2$. Since $\deg \mu_{[2^{\frac14} \cdot]} = [\Q(2^{\frac14}) : \Q] = 4$ by proposition \ref{prop: minimal_polynomials} and since $\chi_{[2^{\frac14} \cdot]}$ is already monic and irreducible, we have that:
                $$\mu_{[2^{\frac14} \cdot]} = \chi_{[2^{\frac14} \cdot]}$$
        \end{example}

        \begin{lemma}[Finitely generated algebraic extensions are finite] \label{lemma: finitely_generated_algebraic_extensions_are_finite}
            Let $k$ be a field. If $k(\alpha_1, ..., \alpha_n)/k$ is an algebraic extension, then it will also be finite. Consequently, finitely generated algebraic extensions can be constructed as towers of simple algebraic extensions as follows:
                $$
                    \begin{tikzcd}
                	{k(\alpha_1, ..., \alpha_n) \cong k(\alpha_1, ..., \alpha_{n - 1})(\alpha_n)} \\
                	\vdots \\
                	{k(\alpha_1, \alpha_2) \cong k(\alpha_1)(\alpha_2)} \\
                	{k(\alpha_1)} \\
                	k
                	\arrow[from=2-1, to=1-1]
                	\arrow[from=3-1, to=2-1]
                	\arrow[from=4-1, to=3-1]
                	\arrow[from=5-1, to=4-1]
                    \end{tikzcd}
                $$
        \end{lemma}
            \begin{proof}
                Let $k_0 := k$ and for every $i \geq 1$, let $k_i := k(\alpha_1, ..., \alpha_i)$. By proposition \ref{prop: minimal_polynomials}, we know that there exist (monic) irreducible polynomials $f_i \in k_{i - 1}[x]$ such that:
                    $$k_i \cong k_{i - 1}[x]/f_i \cong (k_{i - 2}[x]/f_{i - 1})/f_i \cong k_{i - 2}[x]/(f_{i - 1}, f_i)$$
                for every $i \geq 2$. From this, we infer that:
                    $$[k_i : k_{i - 2}] = [k_i : k_{i - 1}] [k_{i - 1} : k_{i - 2}] = \deg f_i \deg f_{i - 1} < +\infty$$
                Since $[k_1 : k] = \deg f_1 < +\infty$ by an application of proposition \ref{prop: minimal_polynomials} to the simple algebraic extension $k_1/k$, we infer inductively from the above that:
                    $$[k_n : k] = [k_n : k_{n - 1}] [k_{n - 1} : k_{n - 2}] ... [k_1 : k] < +\infty$$

                Also, the fact that $k_i \cong k_{i - 1}[x]/f_i \cong (k_{i - 2}[x]/f_{i - 1})/f_i$ implies via proposition \ref{prop: minimal_polynomials} that:
                    $$k(\alpha_1, ..., \alpha_i) =: k_i \cong k_{i - 1}(\alpha_i) \cong k(\alpha_1, ..., \alpha_{i - 1})(\alpha_i)$$
                and so we indeed have a tower of simple algebraic extensions as sought for.
            \end{proof}
        \begin{remark}
            If $\alpha_1, \alpha_2$ are roots over a field $k$ of two monic irreducible polynomials $f_1, f_2 \in k[x]$, then:
                $$k(\alpha_1)(\alpha_2) \cong k(\alpha_1, \alpha_2) \cong k(\alpha_2, \alpha_1) \cong k(\alpha_2)(\alpha_1)$$
            i.e. the order in which on performs the intermediate simple algebraic extensions inside a given finitely generated algebraic extension does not matter.
        \end{remark}
        \begin{example}
            By constructing the extension $\Q(2^{\frac13}, i) \cong \Q(2^{\frac13})(i)$ of $\Q$, one obtains all the $3^{rd}$ roots of $2$ in $\bbC$. The two minimal polynomials are, respectively, $x^3 - 2$ and $x^2 + 1$. 
        \end{example}

        \todo[inline]{Finite algebraic extensions}

        \todo[inline]{Composite fields}

        In order to avoid confusions down the line, it is now a good time to make a slight detour in order to discuss the notion of \say{composita} of field extensions.  
        \begin{proposition}[Composite fields] \label{prop: composite_fields}
            Let $k_1/k$ and $k_2/k$ be two field extensions such that $k_1 \tensor_k k_2 \not \cong 0$. Then, there shall exist a field $k_1 k_2$, called the \textbf{compositum} of $k_1$ and $k_2$ that is minimal amongst all extensions $K/k$ that contain both $k_1/k$ and $k_2/k$ as subextensions.

            Furthermore, $k_1 \tensor_k k_2 \not \cong 0$ if and only if there exists an \say{ambient} field extension $K/k$ admitting both $k_1/k$ and $k_2/k$ as subextensions.
        \end{proposition}
            \begin{proof}
                Since $k_1 \tensor_k k_2 \not \cong 0$, the $k$-algebra $k_1 \tensor_k k_2$ must either be a field or when it is not a field, have a maximal ideal, say $\m$, for which $k_1 k_2 := (k_1 \tensor_k k_2)/\m$ is a field. By lemma \ref{lemma: ring_maps_from_fields_are_injective}, we know that the composite ring maps $k_1 \to k_1 \tensor_k k_2 \to k_1 k_2$ and $k_2 \to k_1 \tensor_k k_2 \to k_1 k_2$ are necessarily injective, and hence $k_1, k_2$ are subfields of $k_1 k_2$. Finally, as $k_1 \tensor_k k_2$ is the pushout of $k_1$ and $k_2$ in the category of commutative rings, it is guaranteed that $k_1 k_2/k$ as above is minimal amongst all extensions $K/k$ that contain both $k_1/k$ and $k_2/k$ as subextensions.

                Now, through remark \ref{remark: extensions_have_the_same_characteristic_as_base}, we see that $k_1 \tensor_k k_2 \cong 0$ if and only if either ring maps $k \to k_1, k \to k_2$ is zero. But these are ring maps with a field, namely $k$, as domains, so they must be injective by lemma \ref{lemma: ring_maps_from_fields_are_injective}, and hence they are zero if and only if $k \cong 0$ or $\chara k \not = \chara k_1, \chara k_2$. The ring $0$ is not a field, so it must be the case that $\chara k \not = \chara k_1, \chara k_2$, in which case there can not exist any field $K$ that contains both $k_1, k_2$ as subfields by remark \ref{remark: extensions_have_the_same_characteristic_as_base}. We have thus proven the contraposition, and hence the original claim.
            \end{proof}
        \begin{remark}
            In a general category $\C$, if a pushout $X \pushout_W Y$ exists, then it will depend on the two morphisms $W \to X$ and $W \to Y$. For us, this implies that the compositum $k_1 k_2/k$ of two field extensions $k_1/k$ and $k_2/k$ makes sense strictly as an extension of $k$, i.e. as an object of ${}^{k/}\Fld$, not as an abstract field on its own, because it depends on the embeddings $k \hookrightarrow k_1$ and $k \hookrightarrow k_2$.
        \end{remark}
        \begin{definition}[Linearly disjoint extensions] \label{def: linearly_disjoint_extensions}
            Two field extensions $k_1/k$ and $k_2/k$ are said to be \textbf{linearly disjoint} if $k_1 \tensor_k k_2 \cong k_1 k_2$ as $k$-algebras.
        \end{definition}
        \begin{lemma}[Linearly disjoint generators] \label{lemma: linearly_disjoint_generators}
            Let $k$ be a field and let $k_1/k, k_2/k$ be two field extensions, say generated by $\{\alpha_i\}_{i \in I}$ and $\{\beta_j\}_{j \in J}$. If $k_1/k, k_2/k$ are linearly disjoint, then $\{\alpha_i\}_{i \in I} \cap \{\beta_j\}_{j \in J} = \varnothing$ and $k_1 k_2$ will thus be generated by $\{\alpha_i\}_{i \in I} \sqcup\{\beta_j\}_{j \in J}$. 
        \end{lemma}
            \begin{proof}
                From proposition \ref{prop: composite_fields}, we know that $k_1 \tensor_k k_2$ is generated, as a $k$-algebra, by the pure tensors $\alpha_i \tensor \beta_j$. Their images in $k_1 k_2$ are $\alpha_i \beta_j$, so if there exists some $\beta_j = \alpha_i$ then $\alpha_i \beta_j = \alpha_i^2 \in k_1$, which makes $\beta_j$ a redundant generator for $k_1 k_2$.
            \end{proof}
        \begin{example}
            If $p \not = q$ are distinct primes, then the extensions $\Q(p^{\frac12})/\Q$ and $\Q(q^{\frac12})/\Q$ will be linearly disjoint.
            
            However, if $m \not = n$ are two not-necessarily-prime distinct square-free integers, then $\Q(m^{\frac12})/\Q$ and $\Q(n^{\frac12})/\Q$ can still be linearly non-disjoint. For instance, take $m := 2$ and $n := 8$. Then actually, we have that $\Q(n^{\frac12})/\Q = \Q(8^{\frac12})/\Q = \Q(2^{\frac12})/\Q = \Q(m^{\frac12})/\Q$, wherein the middle equality occurs because $8^{\frac12} = 2 \cdot 2^{\frac12}$, and $2 \in \Q$.
        \end{example}

        \todo[inline]{Algebraic closures}

        \begin{definition}[Algebraic closures] \label{def: algebraic_closures}
            An \textbf{algebraic closure} of a field $k$ is an algebraic extension $C/k$ such that every algebraic extension $C'/C$ is trivial. A field is said to be \textbf{algebraically closed} if it is isomorphic to an algebraic closure thereof.
        \end{definition}
        \begin{theorem}[Existence and uniqueness of algebraic closures] \label{theorem: existence_and_uniqueness_of_algebraic_closures}
            Every field admits an algebraic closure. Furthermore, all algebraic closures of a given field $k$ are isomorphic to one another as $k$-algebras, so let us only refer to an arbitrary representative $k^{\alg}$ of this isomorphism class as \say{the} algebraic closure of $k$.

            Consequently, every algebraic extension of $k$ is a subextension of $k^{\alg}$.
        \end{theorem}
            \begin{proof}
                See \cite[\href{https://stacks.math.columbia.edu/tag/09GT}{Tag 09GT}]{stacks}.
            \end{proof}
        \begin{proposition}[Algebraic closures contain all roots] \label{prop: algebraic_closures_contain_all_roots}
            Let $k$ be a field. Then, the image of any \underline{non-constant} $f \in k[x]$ under the embedding $k[x] \hookrightarrow k^{\alg}[x]$ will be completely reducible (over $k^{\alg}$) into $\deg f$-many linear factors in the following manner:
                $$f = c \prod_{i = 1}^{n \leq \deg f} (x - \alpha_i)^{e_i}$$
            with $\sum_{i = 1}^n e_i = \deg f$ and for some $c \in k$. The numbers $e_i \in \N$ are called the \textbf{algebraic multiplicities} of the roots $\alpha_i$ of $f$ over $k^{\alg}$.
        \end{proposition}
            \begin{proof}
                Every algebraic extension of $k^{\alg}$ is trivial by definition, so by the definition of algebraic extensions (see definition \ref{def: algebraic_and_transcendental_extensions}), for every maximal ideal $\m \subset k^{\alg}[x]$, we must have that $k^{\alg}[x]/\m \cong k^{\alg}$ or in other words, that we have a short exact sequence:
                    $$0 \to \m \xrightarrow[]{\ker \pi_{\m}} k^{\alg}[x] \xrightarrow[]{\pi_{\m}} k^{\alg} \to 0$$
                for every such $\m$. Corollary \ref{coro: maximal_ideals_of_polynomial_rings_over_fields} tells us that $\m = (f)$ for some irreducible $f \in k^{\alg}[x]$, and proposition \ref{prop: minimal_polynomials} then tells us that $\deg f = [k^{\alg} : k^{\alg}] = 1$. We infer from this that every irreducible $f \in k^{\alg}[x]$ must actually be linear, i.e. $f = x - \alpha$ for some $\alpha \in k^{\alg}$. Via the Euclidean algorithm, we know that every polynomial factors into a product of irreducible polynomials, and so we are done.
            \end{proof}

    \subsection{Properties of algebraic extensions}
        \todo[inline]{Separable extensions}
        
        In light of proposition \ref{prop: algebraic_closures_contain_all_roots}, it is natural to ask the following question:
        \begin{question}
            Let $k$ be a field and $k(\alpha_1, ..., \alpha_n)$ be a finitely generated algebraic extension (equivalently, a finite extension; see corollary \ref{coro: finite_extensions_are_algebraic} and lemma \ref{lemma: finitely_generated_algebraic_extensions_are_finite}). Also, let $f_1, ..., f_n$ denote the minimal polynomials of $\alpha_1, ..., \alpha_n$ respectively. Then, is there a well-defined class of extensions of $k$ in which the image under $k[x] \to k^{\alg}[x]$ of each $f_i$ ($1 \leq i \leq n$) has pairwise distinct roots - and hence all the roots of $f_i$ are necessarily of multiplicity $1$ (this follows also from irreducibility) ?
        \end{question}
        \begin{definition}[Separable extensions] \label{def: separable_extensions} 
            Let $k$ be a field.
            \begin{enumerate}
                \item If a polynomial $f \in k[x]$ has pairwise distinct roots over $k^{\alg}$ (each necessarily of multiplicity $1$) then we will say that $f$ is \textbf{separable}.
                \item An algebraic extension $k'/k$ is said to be \textbf{separable} if and only if the minimal polynomial of every element $\alpha \in k'$ is separable. Elements of $k'$ are thus also said to be \textbf{separable} over $k$.
            \end{enumerate}
        \end{definition}
        \begin{remark}
            A very important distinction to make is, that \textit{it is not true that if $k'/k$ is separable then the minimal polynomial of any $\alpha \in k'$ splits completely into linear factors over $k'$ (cf. definition \ref{def: normal_and_galois_extensions})!} This is because in general, any arbitrarily given separable extension may differ from an algebraic closure. 
        \end{remark}
        \begin{lemma}[A separability criterion for polynomials] \label{lemma: separability_criterion_for_polynomials}
            Let $k$ be a field. A polynomial $f(x) \in k[x]$ is separable if and only if $\left(f(x), \frac{d}{dx} f(x)\right) = k[x]$.
        \end{lemma}
            \begin{proof}
                Without any loss of generality, we can assume that $f(x)$ is monic. By proposition \ref{prop: algebraic_closures_contain_all_roots}, we know that in $k^{\alg}[x]$, we have the factorisation:
                    $$f(x) = \prod_{i = 1}^{\deg f} (x - \alpha_i)$$
                for some $\alpha_i \in k^{\alg}$ which perhaps are pairwise indistinct. Per the Leibniz rule, we then have that:
                    $$\frac{d}{dx} f(x) = \frac{d}{dx} \prod_{i = 1}^{\deg f} (x - \alpha_i) = \sum_{j = 1}^{\deg f} \prod_{i = 1, i \not = j}^{\deg f} (x - \alpha_i)$$

                Now, if $f(x)$ is separable, then the roots $\alpha_i$ are pairwise distinct by definition, and hence $(x - \alpha_j) \nmid \frac{d}{dx} f(x)$ for all $1 \leq j \leq \deg f$ and hence $\frac{d}{dx} f(x)$ is coprime to $f(x)$, and hence these two polynomials generate the unit ideal of $k[x]$.

                Conversely, if $\frac{d}{dx} f(x)$ and $f(x)$ generate the unit ideal of $k[x]$, then $(x - \alpha_j) \nmid \frac{d}{dx} f(x)$ for all $1 \leq j \leq \deg f$. If we assume for the sake of deriving a contradiction that there exist $1 \leq i \not = j \leq \deg f$ for which $\alpha_i = \alpha_j$. But then, because $\frac{d}{dx} f(x) = \sum_{j = 1}^{\deg f} \prod_{i = 1, i \not = j}^{\deg f} (x - \alpha_i)$, our assumption would then imply that there exists some $1 \leq j \leq \deg f$ such that $(x - \alpha_j) \mid \frac{d}{dx} f(x)$, which is absurd. Therefore, $f(x)$ must be separable.
            \end{proof}
        The following is a peculiarity of extensions over positive characteristics. 
        \begin{proposition}[Purely inseparable polynomials] \label{prop: purely_inseparable_polynomials}
            Let $p$ be a prime and $k$ be a field of characteristic $p$. Then, 
        \end{proposition}
            \begin{proof}
                
            \end{proof}
        \begin{definition}[Purely inseparable extensions] \label{def: purely_inseparable_extensions}
            
        \end{definition}
        \begin{example}[Perfections are purely inseparable] \label{example: perfections_are_purely_inseparable}
            
        \end{example}

        \begin{definition}[Separable closures] \label{def: separable_closures}
            
        \end{definition}
        As it turns out, purely inseparable extensions can only be of positive characteristics. 
        \begin{lemma}[No purely inseparable extensions over characteristic $0$] \label{lemma: no_purely_inseparable_extensions_over_char_0}
            Every algebraic extension of a field of characteristic $0$ (and hence every algebraic extension of $\Q$) is separable. Consequently, if $k$ is a field and $\chara k = 0$ then $k^{\sep} \cong k^{\alg}$.
        \end{lemma}
            \begin{proof}
                
            \end{proof}

        \todo[inline]{Splitting fields}

    \subsection{Galois groups and the Galois Correspondence}
        \begin{definition}[Normal and Galois extensions] \label{def: normal_and_galois_extensions}
            
        \end{definition}