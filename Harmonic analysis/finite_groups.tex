\section{Finite groups}
    \begin{remark}[Class equation for the adjoint action] \label{remark: class_equation}
        The following is an important observation that we will be making use of liberally all throughout. 
    
        Let $G$ be a group acting on a set $X$ and let $\pi: X \to X/G$ be the quotient map, given by $\pi(x) := G \cdot x$. In this case, we have the following partition of $X$ into $G$-orbits:
            $$X \cong \coprod_{[x] \in X/G} G \cdot x$$
        where $x, x' \in [x] \pmod{G} \iff x, x' \in G \cdot x$. Let $X^G := \{x \in X \mid \forall g \in G: g \cdot x = x\}$ be the set of \textbf{$G$-fixed points} in $X$, with respect to the given action. Note that if $x \in X^G$ then:
            $$G \cdot x = \bigcup_{g \in G} \{g \cdot x\} = \{x\}$$
        which means that $\pi(X^G) \cong \{*\}$ Now, we can decompose $X$ more granularly as follows:
            $$X \cong X^G \sqcup \coprod_{[x] \in X/G \setminus \pi(X^G)} G \cdot x = X^G \sqcup \coprod_{[x] \in X/G, x \not \in X^G} G \cdot x$$
    
        Now, consider the special case wherein $G$ acts on itself by conjugations, i.e. $G$ is being acted on by $\Ad(G)$, the image of $\Ad: G \to \Aut_{\Grp}(G)$. In this case, we have that $\Stab_G(x) = \Cent_G(x)$ tautologically for all $x \in G$, and $\rmZ(G) = G^{\Ad(G)}$, also tautologically, is nothing but the set of elements $x \in G$ fixed by all $g \in G$. Thus, the partition of $G$ into orbits reads:
            $$G \cong \rmZ(G) \sqcup \coprod_{[x] \in G/\Ad(G)} G/\Cent_G(x)$$
        and note that this is just a bijection of sets, \textit{not a group isomorphism}, since stabilisers are not normal as subgroups in general. When $G$ is finite, the partition above gives rise to the following \textbf{class formula}:
            $$\ord(G) = \ord( \rmZ(G) ) + \sum_{[x] \in G/\Ad(G), x \not \in \rmZ(G)} [G : \Cent_G(x)]$$
    \end{remark}
    
    \subsection{\texorpdfstring{$p$}{}-groups and Sylow subgroups}
        \begin{definition}[$p$-groups] \label{def: p_groups}
            For a prime $p$, a $p$-group is a group whose order is some finite power of $p$.
        \end{definition}
        \begin{lemma}[$p$-subgroups of finite abelian groups] \label{lemma: p_subgroups_of_finite_abelian_groups}
            Let $E$ be a non-trivial finite abelian group of order $n$ and let $p \mid n$ be a prime divisor. Then, there shall exist a subgroup $E' \subset E$ that is of order exactly $p$.
        \end{lemma}
            \begin{proof}
                To see why this is the case, note that by the Decomposition Theorem for finitely generated abelian groups, we know that:
                    $$E \cong \bigoplus_i \Z/p_i^{r_i}$$
                where $p_i$ are prime numbers such that the prime factorisation of $m = \ord(E)$ takes the form:
                    $$m = \prod_i p_i^{r_i}$$
                Each $\Z/p^{r_i}$ admits $\<p_i\>/\<p_i^{r_i}\> \cong \Z/p_i$ as a subgroup of order $p_i$, so we are done.
            \end{proof}
        In example \ref{example: abelian_sylow_subgroups}, we will see also that finite abelian groups $E$ actually admit towers of $p$-subgroups for each $p \mid \ord(E)$, and that such towers have maximal elements called \say{Sylow $p$-subgroups} of $E$. 
        \begin{proposition}[Centres of $p$-groups] \label{prop: centres_of_p_groups}
            For every prime $p$, any non-trivial $p$-group must have a non-trivial centre.
        \end{proposition}
            \begin{proof}
                Let $G$ be a non-trivial $p$-group; set $\ord(G) := p^s$ for some $s \geq 1$. Let $G$ act on itself by conjugation and let us rewrite the class equation in this case into:
                    $$\ord( \rmZ(G) ) = \ord(G) - \sum_{[x] \in G/\Ad(G), x \not \in \rmZ(G)} [G : \Cent_G(x)]$$
                By the Orbit-Stabliser theorem, $[G : \Cent_G(x)] \mid \ord(G) = p^s$ so $[G : \Cent_G(x)] = p^{s_x}$ for some $s_x \in \N_{\leq s}$. We see thus that $p \mid \ord(\rmZ(G))$, and hence $\ord(\rmZ(G)) \geq p$, and the centre $\rmZ(G)$ thereforer can not be trivial.
            \end{proof}

        \begin{lemma}[$p$-group fixed points] \label{lemma: p_group_fixed_points}
            Let $G$ be a $p$-group acting on a finite set $X$. Then:
                $$|X| \equiv |X^G| \pmod{p}$$
            In particular, this means that if $p \mid |X|$ then $|X^G| \equiv 0 \pmod{p}$.
        \end{lemma}
            \begin{proof}
                From the class equation, we infer that:
                    $$|X| \equiv |X^G| \sum_{[x] \in X/G, x \not \in X^G} [G : \Stab_G(x)] \pmod{p}$$
                Since $G$ is a $p$-group, we have that:
                    $$\sum_{[x] \in X/G, x \not \in X^G} [G : \Stab_G(x)] \pmod{p} \equiv 0 \pmod{p}$$
                and so we indeed have that:
                    $$|X| \equiv |X^G| \pmod{p}$$
            \end{proof}
    
        A very basic result in the theory of finite group is one by Lagrange, which tells us that should $G$ be a finite group and $H \leq G$ be a subgroup thereof, then:
            $$\ord(H) \mid \ord(G)$$
        The converse, however, is not true: in general, one does not expect to find a subgroup $H$ of a prescribed order $d \mid \ord(G)$. Fortunately, the next best thing is available, namely that one can still expect $G$ to admit maximal $p$-subgroups. That is to say, if $\ord(G)$ has (positive) prime factorisation:
            $$\ord(G) := \prod_i p_i^{r_i}$$
        then there shall exist subgroups $H_i$ which are of orders $p_i^{r_i}$ and hence maximal amongst the $p_i$-subgroups of $G$. Furthermore, if $H_i, H_i'$ are two maximal $p_i$-subgroups of $G$, then they will even be necessarily conjugate to one another, thus limiting how spreaded out these trees of subgroups of $G$ can be. Finally, it is even known that the number of maximal $p_i$-subgroups of $G$ (so really, the number of conjugacy classes of any given such subgroup) is congruent to $1$ modulo $p_i$.

        \begin{remark}[Modular general linear groups] \label{remark: modular_general_linear_groups}
            Let $p$ be a prime and let $q := p^d$ for some $d \geq 1$, and then consider the finite field $\F_q$ of $q$-elements, obtainable as the splitting field of $x^q - x \in \F_p[x]$.

            Next, consider the group of invertible $n \x n$ matrices with entries coming from $\F_q$, i.e. $\GL_n(\F_q)$. In general, the order of $\GL_n(k)$ for any field $k$ is nothing but the number of possible bases for $k^{\oplus n}$, since an $n \x n$ matrix is invertible if and only if its columns form a basis for $k^{\oplus n}$. Given such a basis:
                $$\{v_1, ..., v_n\} \subset k^{\oplus n}$$
            there are $|k|^n - 1$ ways of choosing $v_1$ since we can choose anything but $0$, $|k|^n - |k|$ to choose $v_2$, etc. and in general, $|k|^n - |k|^{i - 1}$ to choose $v_i$. In total, there are:
                $$(|k|^n - 1)(|k|^n - |k|) ... (|k|^n - |k|^{n - 1}) = \prod_{i = 1}^n (|k|^n - |k|^{i - 1})$$
            ways of choosing bases for $k^{\oplus n}$. We therefore have that:
                $$\ord( \GL_n(\F_q) ) = \prod_{i = 1}^n (q^n - q^{i - 1})$$

            From this, we see that:
                $$\log_p \ord( \GL_n(\F_q) ) = \sum_{i = 1}^n \log_p (q^n - q^{i - 1}) = \sum_{i = 1}^n \left( d(i - 1) + \log_p( q^{n - i - 1} - 1 ) \right) = d \frac{n(n - 1)}{2} + \sum_{i = 1}^n \log_p( q^{n - i - 1} - 1 )$$
            Exponentiating both sides back up will then yield:
                $$\ord( \GL_n(\F_q) ) = p^{ d \frac{n(n - 1)}{2} } \prod_{i = 1}^n (q^{n - i - 1} - 1)$$
            All of the factors $q^{n - i - 1} - 1$ are coprime to $p$, so any Sylow $p$-subgroup of $\GL_n(\F_q)$ must be of order $p^{ d\frac{n(n - 1)}{2} }$, and $\GL_n(\F_q)$ can not have $p'$-subgroups for any other prime $p' \not = p$.
        \end{remark}

        Let us show that any symmetric group can be embedded into some $\GL_n(\F_q)$, and then exploit the fact that every finite group $G$ embeds as a subgroup into the subgroup $S_{\ord(G)}$ if we let $G$ act on itself via left/right-multiplication, in order to show that finite groups embed into modular general linear groups.
        \begin{proposition}[Symmetric groups as matrix groups] \label{prop: symmetric_groups_as_matrix_groups}
            Fix a positive integer $n$ along with some finite field $\F_q$. Then, the symmetric group $S_n$ admits a faithful representation:
                $$\rho: S_n \to \GL_n(\F_q)$$
            which realises $S_n$ as a subgroup of $\GL_n(\F_q)$.
        \end{proposition}
            \begin{proof}
                Choose a basis $\{v_1, ..., v_n\} \subset \F_q^{\oplus n}$ and then let the elements $\sigma \in S_n$ act on this basis by:
                    $$\rho(\sigma) \cdot v_i := v_{\sigma(i)}$$
                i.e. by permuting the basis vectors. This clearly extends to an $\F_q$-linear action on $\F_q^{\oplus n}$, so the only left to show is that $\rho$ is injective. For this, let us note firstly that because basis vectors are non-zero by definition, we have that:
                    $$v_{\sigma(i)} \not = 0$$
                for all $1 \leq i \leq n$. Next, because if $v = \sum_{i = 1}^n a_i v_i$ then $\rho(\sigma) \cdot v = \sum_{i = 1}^n a_i v_{\sigma(i)}$, which vanishes if and only if $a_i = 0$ for all $1 \leq i \leq n$, since each $v_{\sigma(i)}$ is a basis vector, and hence if and only if $v = 0$. This proves injectivity.
            \end{proof}
        \begin{corollary}[Finite groups as matrix groups] \label{coro: finite_groups_as_matrix_groups}
            Any finite group $G$ has a faithful representation:
                $$\rho_G: G \to \GL_{\ord(G)}(\F_q)$$
            for any finite field $\F_q$.
        \end{corollary}
            \begin{proof}
                Embed $G$ as a subgroup into $S_{\ord(G)}$ using Cayley's Theorem, and then apply proposition \ref{prop: symmetric_groups_as_matrix_groups}.
            \end{proof}

        \begin{proposition}[Existence of $p$-subgroups (Cauchy's Theorem)] \label{prop: existence_of_p_subgroups}
            If $p$ is a prime number that divides the order of some finite group $G$, then $G$ will admit a subgroup of order $p$ (and hence said subgroup is necessarily isomorphic to $\Z/p$).
        \end{proposition}
            \begin{proof}
                Let $n := \ord(G)$.

                \todo[inline]{Not done.}
            \end{proof}

        In honour of the Norwegian mathematician who discovered the remarkable results listed at the beginning, Peter Ludvig Meidell Sylow, maximal $p_i$-subgroups of $G$ are afforded a special name:
        \begin{definition}[Sylow subgroups] \label{def: sylow_subgroups}
            A \textbf{Sylow $p$-subgroup} of a finite group $G$ is a subgroup $H \leq G$ that is maximal\footnote{This maximality is only well-defined thanks to proposition \ref{prop: existence_of_p_subgroups}.} amongst all $p$-subgroups of $G$.
        \end{definition}
        \begin{lemma}[A maximality criterion for $p$-subgroups] \label{lemma: maximality_criterion_for_p_subgroups}
            Let $G$ be a finite group and $p$ be a prime. If $H \leq G$ is a $p$-subgroup, then $H$ will be a Sylow $p$-subgroup of $G$ if and only if $\gcd( [G : H], p ) = 1$. 
        \end{lemma}
            \begin{proof}
                Set $\ord(H) := p^n$.
            
                Suppose firstly that $H$ is a Sylow $p$-subgroup. By Lagrange's Theorem, we know that $[G : H] = \frac{\ord(G)}{\ord(H)} = \frac{\ord(G)}{p^n}$, and since $n$ is as large as possible, $[G : H]$ therefore does not contain any power of $p$ as a factor (i.e. there does not exist any $m \geq 1$ such that $p^m \mid [G : H]$), and hence $\gcd( [G : H], p ) = 1$.

                Conversely, suppose that $\gcd([G : H], p) = 1$. This means that there does not exist any $m \geq 1$ such that $p^m \mid [G : H]$. By Lagrange's Theorem again, we know that $\ord(G) = [G : H] \ord(H) = [G : H] p^n$. We see thus that $n$ as large as possible so that $p^n \mid \ord(G)$, and hence $H$ is a Sylow $p$-subgroup by definition. 
            \end{proof}
        \begin{example}[Abelian Sylow $p$-subgroups] \label{example: abelian_sylow_subgroups}
            If $E$ is a finite abelian group then by the Decomposition Theorem for finitely generated abelian groups, we know that:
                $$E \cong \bigoplus_i \Z/p_i^{r_i}$$
            and hence the Sylow $p_i$-subgroups of $E$ shall be isomorphic to $\Z/p_i^{r_i}$. This, in turn, implies that for finite abelian groups, Sylow $p$-subgroups not only exist but for each $p \mid \ord(E)$, there is only one such subgroup up to isomorphisms.
        \end{example}
        
        \begin{lemma}[Modular general linear groups admit Sylow subgroups] \label{lemma: modular_general_linear_groups_admit_sylow_subgroups}
            Let $n$ be a positive integer and $\F_q$ be any finite field. Then $\GL_n(\F_q)$ admits $U^+_n(\F_q)$, the group of upper-triangular matrices in $\GL_n(\F_q)$ whose diagonal entries are $1$.
        \end{lemma}
            \begin{proof}
                
            \end{proof}
        The fact that $\GL_n(\F_q)$ admits a Sylow $p$-subgroup is important for showing that Sylow $p$-subgroups exist also for general finite groups. Our argument will rely on the fact that any finite group can be embedded as a subgroup into some $\GL_n(\F_q)$, for some $n \gg 1$. 
        \begin{theorem}[Sylow I: Existence of Sylow subgroups] \label{theorem: sylow_1_existence_of_sylow_subgroups}
            Let $G$ be a finite group and $p$ be a prime divisor of $\ord(G)$. Let $s \in \N_{\geq 1}$ be the largest positive integer such that $p^s \mid \ord(G)$. Then, $G$ shall admit a Sylow $p$-subgroup of order $p^s$.
        \end{theorem}
            \begin{proof}
                Let $p$ be a prime that divides $n := \ord(G)$ and let $q := p^d$ for any $d \geq 1$. If $G$ is already a $p$-group then $G$ will admit itself as a Sylow $p$-subgroup\footnote{As we do not require Sylow subgroups to be proper.}, so from now on let us suppose that $G$ is not a $p$-group.
                
                We now know by lemma \ref{lemma: modular_general_linear_groups_admit_sylow_subgroups} that any $\GL_n(\F_q)$ admits a Sylow $p$-subgroup, say $B$. Firstly $B \cap G$ is a subgroup of both $B$ and $G$ and as such, $\ord(B \cap G)$ will divide both $p^b := \ord(B)$ and $n := \ord(G)$. From this, one sees that $\ord(B \cap G)$ must be of the form $p^s$ for some $s \leq b$. Now, we know by lemma \ref{lemma: maximality_criterion_for_p_subgroups} that if $\gcd( [G : B \cap G], p ) = 1$ then $B \cap G$ will automatically be maximal amongst $p$-subgroups of $G$, hence a Sylow $p$-subgroup by definition. As $G$ is recognisable as a subgroup of $\GL_n(\F_q)$ (see corollary \ref{coro: finite_groups_as_matrix_groups}), we have that:
                    $$\ord(G) \mid \ord(\GL_n(\F_q)) = p^{d \frac{n(n - q)}{2}} \prod_{i = 1}^n (p^{d(n - i - 1)} - 1)$$
                (see remark \ref{remark: modular_general_linear_groups} for a derivation of the RHS). Since $G$ is not a $p$-group, $\ord(G) > 1$ (the trivial group is a $p$-group of order $p^0$ for any prime $p$) can not divide $q^{\frac{n(n - q)}{2}}$, meaning that:
                    $$\ord(G) \mid \prod_{i = 1}^n (p^{d(n - i - 1)} - 1)$$
                At the same time, we have by Langrange's Theorem that:
                    $$[G : B \cap G] \mid \ord(G)$$
                and hence:
                    $$[G : B \cap G] \mid \prod_{i = 1}^n (p^{d(n - i - 1)} - 1)$$
                We then have that:
                    $$\gcd( [G : B \cap G], p ) \leq 1 \gcd( \prod_{i = 1}^n (p^{d(n - i - 1)} - 1), p ) = 1$$
                and since GCDs are $\geq 1$ by definition, the above implies that:
                    $$\gcd( [G : B \cap G], p ) = 1$$
                which as mentioned above, implies that $B \cap G$ is a Sylow $p$-subgroup of $G$.
            \end{proof}

        Now, let us see how there can be many Sylow $p$-subgroups for the same $p$, particularly when those subgroups are non-normal.
        \begin{lemma}[Subgroups of $p$-groups] \label{lemma: subgroups_of_p_groups}
            If $G$ is a $p$-group, say of order $\ord(G) := p^s$ for some $s \in \N$, then for every $r \in \N_{\leq s}$, there shall exist at least one subgroup $H \leq G$ such that $\ord(H) = p^r$.
        \end{lemma}
            \begin{proof}
                
            \end{proof}
        \begin{theorem}[Conjugacies of Sylow subgroups: Sylow II] \label{theorem: sylow_2_conjugacies_of_sylow_subgroups}
            Let $G$ be a finite group. Then, for any $p \mid \ord(G)$, all the Sylow $p$-subgroups of $G$ are conjugate to one another. 
        \end{theorem}
            \begin{proof}
                From the proof of theorem \ref{theorem: sylow_1_existence_of_sylow_subgroups}, we know that every Sylow $p$-subgroup of $G$ arises as the intersection of a subgroup of $G$ with $U^+(\F_q) \leq \GL_n(\F_q)$ (here, $n := \ord(G)$). If $N^+(\F_q) \leq \GL_n(\F_q)$ denotes the subgroup of strictly upper triangular matrices, then $U^+(\F_q) = 1 + N^+(\F_q)$. If $\GL_n(\F_q)$ acts on itself via conjugations, then observe that it  will act transitively on both the subsets $1$ and $N^+(\F_q)$. As such, when we conjugate only by elements of $G$, we shall see that every Sylow $p$-subgroup of $G$ is conjugate to one another.
            \end{proof}

        \begin{theorem}[Number of Sylow subgroups: Sylow III] \label{theorem: sylow_3_number_of_sylow_subgroups}
            Let $G$ be a finite group. Then, for any $p \mid \ord(G)$, the number of Sylow $p$-subgroups of $G$ is $\equiv 1 \pmod{p}$.
        \end{theorem}
            \begin{proof}
                From the proof of theorem \ref{theorem: sylow_2_conjugacies_of_sylow_subgroups}, we know now that this is simply a matter of counting the number of conjugacy class of $U^+(\F_q)$ that intersect non-trivially with $G$.
            \end{proof}

    \subsection{Solvability and some finite Galois theory}

    \subsection{Characters theory}