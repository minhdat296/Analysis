\section{Normed spaces}
    Throughout, $K$ shall be a complete archimedean field, so either $\R$ or $\bbC$, and the absolute value on $K$ will be denoted by $\abs{-}$. All vector spaces, unless stated otherwise, shall be $K$-vector spaces.

    \begin{convention}
        To conform to a traditional convention in the theory of integral equations, in calculus of variations, and in probability theory, the arguments of functions whose inputs are linear functionals will be placed between square brackets $[]$ as opposed to the usual parentheses $()$. For example, if $V, W$ are vector spaces and $T: V \to W$ is a linear map then the evaluation of the dual map $T^*: W^* \to V^*$ on functionals $\psi \in W^*$ will be written:
            $$T[\psi^*]$$
        instead of $T(\psi^*)$. This may also help with readability in certain cases. 
    \end{convention}
 
    \subsection{Generalities on topological vector spaces}
        \begin{remark}[Quotient norms] \label{remark: quotient_norms}
            Let $\pi: E \to Q$ be a continuous quotient map of normed spaces. Elements of $\bar{x} \in Q$ (for each of which there is an element $x \in E$, due to $\pi$ being surjective) can be written as cosets:
                $$\bar{x} := x + \ker \pi$$
            for any choice of representative $x \in \pi^{-1}(\bar{x})$. From this and from the fact that $Q$ has the finest possible topology such that $\pi$ is continuous, one sees that:
                $$\norm{\bar{x}}_Q := \inf_{y \in \ker \pi} \norm{x - y}_E = \dist(x, \ker \pi)$$
        \end{remark}
        \begin{lemma}[(Co)limits of normed spaces]
            The category $K\-\Vect_{\norm{-}}$ has the following (co)limits:
            \begin{enumerate}
                \item epimorphisms,
                \item 
            \end{enumerate}
        \end{lemma}
            \begin{proof}
                \begin{enumerate}
                    \item 
                    \item 
                \end{enumerate}
            \end{proof}

        \begin{definition}[Baire spaces] \label{def: baire_spaces}
            A \textbf{Baire space} is a topological space in which countable intersections of dense open subsets are once again dense.
        \end{definition}
        \begin{definition}[Nowhere dense and meagre subsets] \label{def: nowhere_dense_and_meagre_subsets}
            A subset $M$ of a topological space $X$ is said to be:
            \begin{itemize}
                \item \textbf{nowhere dense} if $\overline{M}$ has an empty interior,
                \item \textbf{meagre} if it can be written as a countable union of nowhere dense subsets; clearly, nowhere dense subsets of $X$ are also meagre. 
            \end{itemize}
        \end{definition}
        \begin{lemma}[Nowhere subsets are complements of dense subsets] \label{lemma: nowhere_dense_subsets_are_complements_of_dense_subsets}
            Let $X$ be a topological space. A subset $M \subseteq X$ is then nowhere dense if and only if $S := X \setminus \overline{M}$ is dense. Equivalently, a subset $S \subseteq X$ is dense if and only if $X \setminus S$ is nowhere dense.
        \end{lemma}
            \begin{proof}
                If $M \subseteq X$ is nowhere dense then by definition, $(\overline{M})^{\circ} = \varnothing$ and hence we have that:
                    $$S := X \setminus \overline{M} = X \setminus ( (\overline{M})^{\circ} \cup \del \overline{M} ) = X \setminus \del \overline{M} = X \setminus \overline{\del \overline{M}}$$
                From this, we see that:
                    $$\overline{S} = \overline{X \setminus \overline{\del \overline{M}}} = X \setminus (\del \overline{M})^{\circ} = X \setminus \varnothing = X$$
                i.e. $S$ is dense by definition.
                
                Conversely, if $S := X \setminus \overline{M}$ is dense, then:
                    $$\overline{ X \setminus \overline{M} } = X$$
                At the same time, we have that:
                    $$\overline{ X \setminus \overline{M} } = X \setminus M^{\circ} = X \setminus (\overline{M})^{\circ}$$
                Thus, $(\overline{M})^{\circ} = \varnothing$ necessarily, i.e. $M$ is nowhere dense.
            \end{proof}
        
        The lemma above allows for some flexibility with the definition of Baire spaces.
        \begin{lemma}[Baire spaces via meagre subsets] \label{lemma: baire_spaces_via_meagre_subsets}
            An equivalent characterisation of Baire spaces is as follows: a topological space $X$ is Baire if and only if countable unions of meagre closed subsets of $X$ are once more meagre. 
        \end{lemma}
            \begin{proof}
                Suppose firstly that $X$ is Baire and consider some countable union $M := \bigcup_{n \in \N} M_n$ of meagre closed subsets $M_n \subseteq X$. By lemma \ref{lemma: nowhere_dense_subsets_are_complements_of_dense_subsets}, we know that there exist dense open subsets $U_n \subseteq X$ using which we can write $M_n := X \setminus U_n$. We then have that:
                    $$M := \bigcup_{n \in \N} M_n = \bigcup_{n \in \N} (X \setminus U_n) = X \setminus \bigcap_{n \in \N} U_n$$
                As $X$ is a Baire space, $U := \bigcap_{n \in \N} U_n$ must be dense by virtue of being a countable intersection of dense subsets of $X$, and by lemma \ref{lemma: nowhere_dense_subsets_are_complements_of_dense_subsets}, the complement $M = X \setminus U$ must therefore be meagre.

                Conversely, suppose that countable unions $M := \bigcup_{n \in \N} M_n$ of meagre closed subsets $M_n \subseteq X$ are once more meagre. By lemma \ref{lemma: nowhere_dense_subsets_are_complements_of_dense_subsets}, we know that $X \setminus M$ is dense, and we can also find dense open subsets $U_n \subseteq X$ such that $M_n := X \setminus U_n$. Then, consider the following:
                    $$X \setminus M = X \setminus \bigcup_{n \in \N} (X \setminus U_n) = X \setminus \bigcap_{n \in \N} U_n$$
                Since $M$ is meagre, $\bigcap_{n \in \N} U_n$ must be dense by lemma \ref{lemma: nowhere_dense_subsets_are_complements_of_dense_subsets}, and by varying the meagre subsets $M_n$, we shall get that the intersections of any countable collection of dense subsets of $X$ is once more dense, hence $X$ is Baire by definition.
            \end{proof}
        \begin{remark}[Spaces of category I and of category II ?]
            In the literature (see e.g. \cite[Definition, p. 37]{litvak_functional_analysis_notes}), spaces of so-called \say{category I} and \say{category II} are also mentioned. A topological space is of category I if it is meagre, while it is of category II if it is not of category I, i.e. non-meagre.

            Through lemma \ref{lemma: baire_spaces_via_meagre_subsets}, one sees that a space of category II is nothing but a Baire space, and thus a space of category I is any non-Baire space.

            We will not be referring to topological spaces as belonging either to the category I or category II, not least because this gives the impression that there are categories of topological spaces (in the sense of category theory) called \say{I} and \say{II}, but also because we would like to avoid confusion with the Baire Category Theorems I and II (see remark \ref{remark: baire_category_theorems_1_and_2}).
        \end{remark}
            
        \begin{theorem}[The Baire Category Theorem] \label{theorem: baire_category}
            (Cf. \cite[\href{https://stacks.math.columbia.edu/tag/0CQN}{Tag 0CQN}]{stacks}) Assuming the Axiom of Choice along with ZF, every locally compact\footnote{\cite{stacks} uses the term \say{quasi-compact} which is more prevalent in algebraic geometry due to its French origin.} and Hausdorff topological space is Baire.
            
            Without the Axiom of Choice but still with ZF, every \underline{separable}, locally compact, and Hausdorff topological space is Baire.
        \end{theorem}
            \begin{proof}
                Choose a countable collection $\{U_n\}_{n \geq 0}$ of dense open proper subset of a locally compact and Hausdorff space $X$, and since the set of open subsets of $X$ is partially ordered by inclusion, we can assume without any loss of generality that $U_n \supset U_{n + 1}$ for all $n \geq 0$; note that one can not assume the other way around, i.e. that $U_n \subset U_{n + 1}$, since it is not always guaranteed that there may exist a larger dense proper subset containing a given dense proper subset. Given any $x \in X$, let us show that $x \in \overline{\bigcap_{n \geq 0} U_n}$, which amounts to showing that given any open neighbourhood $B_x \ni x$, one has that $B_x \cap \bigcap_{n \geq 0} U_n \not = \varnothing$. 
                
                To begin, note that since each $U_n$ is dense, we have that $B_x \cap U_n \not = \varnothing$. Because $U_n \supset U_{n + 1}$ for each $n \geq 1$ per out assumption above, this implies that this gives rise to a descending chain of non-empty open subsets of $X$ as follows:
                    $$B_x \cap (U_0 \setminus U_1) \supset B_x \cap (U_1 \setminus U_2) \supset ...$$
                For each $n \geq 0$, the set $B_x \cap (U_n \setminus U_{n + 1})$ is an open neighbourhood of $x$. By the fact that $X$ is locally compact, there must exist non-empty compact subsets:
                    $$D_n \subset B_x \cap (U_n \setminus U_{n + 1})$$
                for every $n \geq 0$. Since $X$ is locally compact, the compact subsets $D_n$ are closed. Now, we claim that it is possible to choose the non-empty compact subsets $D_n$ such that:
                    $$\bigcap_{n \geq 0} D_n \not = \varnothing$$
                Since $B_x \cap (U_n \setminus U_{n + 1}) \supset B_x \cap (U_{n + 1} \setminus U_{n + 2})$ for all $n \geq 0$, we have that $(B_x \cap (U_n \setminus U_{n + 1})) \cap (B_x \cap (U_{n + 1} \setminus U_{n + 2})) = B_x \cap (U_{n + 1} \setminus U_{n + 2})$, and hence:
                    $$D_n \cap D_{n + 1} \subset B_x \cap (U_{n + 1} \setminus U_{n + 2}) = (B_x \cap U_{n + 1}) \setminus (B_x \cap U_{n + 2})$$
                which then implies that:
                    $$D_n \cap D_{n + 1} \subset B_x \cap U_{n + 1} \not = \varnothing$$
                and then:
                    $$X \setminus (D_n \cap D_{n + 1}) \supset X \setminus (B_x \cap U_{n + 1}) = (X \setminus B_x) \cup (X \setminus U_{n + 1})$$
                Since $X$ is Hausdorff, $D_n \cap D_{n + 1}$ is compact, and hence closed, and so $X \setminus (D_n \cap D_{n + 1})$ is open, which means that $(X \setminus (D_n \cap D_{n + 1}))^{\circ} = X \setminus (D_n \cap D_{n + 1})$. At the same time, we have by lemma \ref{lemma: nowhere_dense_subsets_are_complements_of_dense_subsets} that, because $U_{n + 1}$ is dense, its complement is nowhere dense, and so:
                    $$\left( (X \setminus B_x) \cup (X \setminus U_{n + 1}) \right)^{\circ} = (X \setminus B_x)^{\circ} \cup \varnothing = X \setminus \overline{B_x}$$
                Since $X$ is Hausdorff, one can always choose the open neighbourhood $B_x \ni x$ such that $X \setminus \overline{B_x} \not = X$. In that case, we would have that:
                    $$X \setminus (D_n \cap D_{n + 1}) \not \supset X$$
                and hence:
                    $$D_n \cap D_{n + 1} \not = \varnothing$$
                We then have that:
                    $$\bigcap_{n \geq 0} D_n \not = \varnothing$$
                as needed.
                    
                Now, note that:
                    $$D_n \subset B_x \cap (U_n \setminus U_{n + 1}) = (B_x \cap U_n) \setminus (B_x \cap U_{n + 1})$$
                and hence:
                    $$D_n \subset B_x \cap U_n$$
                From this, we infer that:
                    $$\varnothing = \bigcap_{n \geq 0} D_n \subset \bigcap_{n \geq 0} B_x \cap U_n = B_x \cap \bigcap_{n \geq 0} U_n$$
                and hence $\bigcap_{n \geq 0} U_n$ is indeed dense inside $X$.
            \end{proof}
        \begin{remark}[There are two ? + a comment on the proof] \label{remark: baire_category_theorems_1_and_2}
            In the literature (cf. e.g. \cite[Theorem 2.2.2]{litvak_functional_analysis_notes}), theorem \ref{theorem: baire_category} is usually referred to as the Baire Category Theorem II. The Baire Category Theorem I is the specialisation of the Theorem II to (complete) metric spaces. In practice, we will usually just be making use of the Theorem I, though we have chosen to prove the Baire Category Theorem II instead to highlight the fact that it is a purely topological assertion.
        \end{remark}
        \begin{remark}
            Theorem \ref{theorem: baire_category} will be used for proving theorems \ref{theorem: uniform_boundedness}, \ref{theorem: open_mapping}, and \ref{theorem: closed_graph}, so it is important that we discuss it first of all.
        \end{remark}

        \begin{lemma}[Suprema of averages] \label{lemma: suprema_of_averages}
            Let $X, Y$ be normed spaces and let $T: X \to Y$ be a continuous linear map between them. Fix $\e > 0$ and a centre $x \in X$. Then:
                $$\sup_{x' \in \B_{\e}(x)} \norm{T(x)}_Y \geq \e \norm{T}_{\Hom_{K, \cont}(X, Y)}$$
        \end{lemma}
            \begin{proof}
                Since every point $x' \in \B_{\e}(x)$ is of distance $< \e$ away from $x$, taking a supremum over all such points is the same as taking a supremum over all translations $x \pm \xi$, with $\xi \in \B_{\e}(0)$; essentially, we are taking $\xi := x' - x$. Thus, consider the following:
                    $$
                        \begin{aligned}
                            & \sup_{x' \in \B_{\e}(x)} \norm{T(x)}_Y
                            \\
                            = & \sup_{\xi \in \B_{\e}(0)} \max\left\{ \norm{T(x \pm \xi)}_Y \right\}
                            \\
                            \geq & \sup_{\xi \in \B_{\e}(0)} \frac12\left( \norm{T(x + \xi)}_Y + \norm{T(x - \xi)}_Y \right) \: \text{(maxima are larger than or equal to averages)}
                            \\
                            \geq & \sup_{\xi \in \B_{\e}(0)} \norm{T(\xi)}_Y \: \text{(triangle inequality)}
                            \\
                            = & \sup_{\xi \in \B_{\e}(0)} \norm{\xi}_X \norm{T}_{\Hom_{K, \cont}(X, Y)}
                            \\
                            = & \e \norm{T}_{\Hom_{K, \cont}(X, Y)}
                        \end{aligned}
                    $$
            \end{proof}
        \begin{theorem}[Uniform boundedness] \label{theorem: uniform_boundedness}
            Let $X$ be a Banach space and $Y$ be a normed space, and let $\{T_n\}_{n \geq 0}$ be a sequence in $\Hom_{K, \cont}(X, Y)$.
            \begin{enumerate}
                \item \textbf{(\textit{Na\"ive} uniform boundedness):} If the sequence is \textbf{pointwise bounded} in the sense that:
                    $$\forall x \in X: \sup_{n \geq 0} \norm{ T_n(x) }_Y < +\infty$$
                then the sequence will automatically be uniformly bounded, in the sense that:
                    $$\sup_{n \geq 0} \norm{T_n}_{\Hom_{K, \cont}(X, Y)} < +\infty$$
                \item \textbf{(Strong uniform boundedness):} In fact, the same assertion remains true should the sequence $\{T_n\}_{n \geq 0}$ be pointwise bounded only on a non-meagre subset of $X$. 
            \end{enumerate}
        \end{theorem}
            \begin{proof}
                \begin{enumerate}
                    \item Suppose for the sake of deriving a contradiction that $\{T_n\}_{n \geq 0}$ is \textit{not} uniformly bounded. Without any loss of generality, suppose also that $\norm{T_n}_{\Hom_{K, \cont}(X, Y)} \geq \frac{1}{\delta^n}$ for every $n \geq 0$ and some $\delta > 1$. We will now attempt to show that our initial assumption will lead to a contradiction by showing that $\sup_{n \geq 0} \norm{T_n}_{\Hom_{K, \cont}(X, Y)}$ must now be bounded above by a finite number. By lemma \ref{lemma: suprema_of_averages}, we know that:
                        $$\sup_{n \geq 0} \sup_{\xi \in \B_{\e}(0)} \norm{T_n(\xi)}_Y \geq \sup_{n \geq 0} \e \norm{T_n}_{\Hom_{K, \cont}(X, Y)} \geq \sup_{n \geq 0} \e \frac{1}{\delta^n} = \e$$
                    for all $\e > 0$. Now, observe that because $\{T_n\}_{n \geq 0}$ is pointwise bounded, $\sup_{n \geq 0} \norm{T_n(\xi)}_Y$ is finite, and because each $T_n$ is continuous and hence bounded, each $\sup_{\xi \in \B_{\e}(0)} \norm{T_n(\xi)}_Y$ is also finite. Thus:
                        $$\sup_{n \geq 0} \sup_{\xi \in \B_{\e}(0)} \norm{T_n(\xi)}_Y < +\infty$$
                    but at the same time, this contradicts what we have above, which is that:
                        $$\forall \e > 0: \sup_{n \geq 0} \sup_{\xi \in \B_{\e}(0)} \norm{T_n(\xi)}_Y \geq \e$$
                    Therefore, $\{T_n\}_{n \geq 0}$ is uniformly bounded.
                    \item This version relies on theorem \ref{theorem: baire_category}. 
                \end{enumerate}
            \end{proof}
        \begin{corollary}[The category of Banach spaces is self-enriching] \label{coro: hom_between_banach_spaces_are_banach_spaces}
            If $X, Y$ are Banach spaces then $\Hom_{K, \cont}(X, Y)$ will also be a Banach space.
        \end{corollary}
            \begin{proof}
                
            \end{proof}

        \begin{theorem}[Open Mapping Theorem: continuous linear maps are open] \label{theorem: open_mapping}
            
        \end{theorem}
            \begin{proof}
                
            \end{proof}

        \begin{theorem}[Closed Graph Theorem: continuous linear maps reflect limits] \label{theorem: closed_graph}
            
        \end{theorem}
            \begin{proof}
                
            \end{proof}

    \subsection{The Hahn-Banach Theorem and duality}
        One fundamental problem in the theory of normed spaces is the question of whether the dual space may actually just be trivial. We know this to not be true always (e.g. for any measure space $(X, \mu)$, the spaces $L^p(X, \mu)$ and $L^q(X, \mu)$ are both infinite-dimensional - hence non-trivial - and are dual if $\frac1p + \frac1q = 1$) and in fact, as the Hahn-Banach Theorem will show, the dual space tends to be quite large, provided that the normed is \say{controlled} somehow.

        For convenience, let us begin by introducing the following terminology:
        \begin{definition}[Sublinearity] \label{def: sublinearity}
            A function $\rho: E \to \R$ on a vector space $E$ is said to be \textbf{sublinear} if and only if:
            \begin{itemize}
                \item $\rho$ satisfies the triangle inequality, i.e.:
                    $$\forall x, y \in X: \rho(x + y) \leq \rho(x) + \rho(y)$$
                \item $\rho$ preserves scalar multiplication, i.e.:
                    $$\forall \lambda \in K: \forall x \in E: \rho(\lambda x) = \lambda \rho(x)$$
            \end{itemize}
        \end{definition}
        \begin{example}
            Norms and semi-norms are sublinear. 
        \end{example}
        \begin{definition}[Dominance] \label{def: dominance}
            A real-valued function $\rho: X \to \R$ on a set $X$ is said to \textbf{dominant} another such function $f: X \to \R$ if:
                $$\forall x \in X: f(x) \leq \rho(x)$$
            In such a situation, we shall write:
                $$f \leq \rho$$
        \end{definition}

        The following lemma is very important.
        \begin{lemma}[Sublinear functions extend finitely] \label{lemma: sublinear_functions_extend_finitely}
            Let $K := \R$ and let $E$ be an $\R$-vector space. Let $\rho: E \to \R$ be a sublinear function that dominates an $\R$-linear functional:
                $$\varphi_0: E_0 \to \R$$
            defined on some subspace $E_0 \subseteq E$. Then, there shall exist an extension:
                $$\varphi_1: E_1 \to \R$$
            (i.e. $\varphi_1|_{E_0} = \varphi_0$) that remains linear and dominated by $\rho$, where $E_1 \subseteq E$ is any subspace such that $\dim E_1/E_0 < +\infty$. 
        \end{lemma}
            \begin{proof}
                We can prove this lemma by proving that for any $x_1 \in E \setminus E_0$, there shall exist an extension $\varphi_1: E_0 \oplus \R x_1 \to \R$ that remains linear and dominated by $\rho$. One can then repeat the process $\dim E_1/E_0$ times to get the full assertion. To this end, we shall need to specify how the expressions of the kind below are given:
                    $$\varphi_1( x + \lambda x_1 )$$
                for all $x \in E$ and $\lambda \in \R$. Consider, then, the following, which holds due to definition \ref{def: sublinearity}:
                    $$\rho(x + y) = \rho( x + \lambda x_1 - \lambda x_1 + y ) \leq \rho(x + \lambda x_1) + \rho(-\lambda x_1 + y)$$
                for all $x, y \in E$ and all $\lambda \in \R$. That $\varphi_0$ is linear and that $\varphi_0 \leq \rho$ together tell us that:
                    $$\varphi_0(x) + \varphi_0(y) = \varphi_0(x + y) \leq \rho(x + y)$$
                and hence:
                    $$\varphi_0(x) + \varphi_0(y) \leq \rho(x + \lambda x_1) + \rho(-\lambda x_1 + y)$$
                for all $x, y \in E_0$. Notice, then, that if we were to let:
                    $$\varphi_1(x + \lambda x_1) := \varphi_0(x) + \lambda \alpha_1$$
                for some choice of $\alpha_1 \in \R$ then because we would like:
                    $$\varphi_1(x + \lambda x_1) \leq \rho(x + \lambda x_1)$$
                we will have to choose $\alpha_1$ to be such that:
                    $$\varphi_0(x) + \lambda \alpha_1 \leq \rho(x + \lambda x_1) \iff \lambda \alpha_1 \leq \rho(x + \lambda x_1) - \varphi_0(x)$$
                for all $x \in E$. Since we only need to specify $\alpha_1 := \varphi_1(x_1)$, we can simply set $\lambda := 1$ and then choose:
                    $$\alpha_1 := \inf_{x \in E} \left( \rho(x + x_1) - \varphi_0(x) \right)$$
            \end{proof}
        \begin{remark}[Some comments about the proof]
            One key detail about lemma \ref{lemma: sublinear_functions_extend_finitely} (and hence also theorem \ref{theorem: hahn_banach}, which depends on said lemma) is that it relies crucially on everything being defined over a totally ordered archimedean field - so that theidea that a sublinear function can dominate a functional would even make sense - and on said archimedean field being complete, so that we can consider infima. 
        \end{remark}
        The Hahn-Banach Theorem tells us that $\R$-linear functionals actually extend all the way by a certain limiting procedure. 
        \begin{theorem}[Hahn-Banach: non-triviality of dual spaces] \label{theorem: hahn_banach}
            Let $K := \R$ and let $E$ be an $\R$-vector space. Let $\rho: E \to \R$ be a sublinear function that dominates an $\R$-linear functional:
                $$\varphi_0: E_0 \to \R$$
            defined on some subspace $E_0 \subseteq E$. Then, there shall exist an extension:
                $$\varphi_{\infty}: E \to \R$$
            (i.e. $\varphi_{\infty}|_{E_0} = \varphi_0$) that remains linear and dominated by $\rho$.

            Since $\varphi_{\infty} \in E^*$, this means that the dual space $E^*$ is non-trivial, since it is known that by picking $E_0$ to be finite-dimensional, one can always be guaranteed that there is some non-trivial $\varphi_0 \in E_0^*$ (and hence the extension $\varphi_{\infty}$ is also non-trivial).
        \end{theorem}
            \begin{proof}
                By lemma \ref{lemma: sublinear_functions_extend_finitely}, we know that there is an ascending chain of subspaces of $E$:
                    $$E_0 \subset E_1 \subset E_2 \subset ...$$
                which is possibly non-terminating and is such that $\dim E_{i + 1}/E_i = 1$ for all $i \in \N$, and each term $E_i$ comes equipped with a linear extension $\varphi_i: E_i \to \R$ that is dominated by the given sublinear function $\rho$. It is clear that by construction, we also have that:
                    $$\varphi_{i + 1|_{E_i}} = \varphi_i$$
                and so we have a filtered diagram $\{(E_i, \varphi_i)\}_{i \in \N}$ in the slice category $\R\-\Vect_{/\R}$. The category $\R\-\Vect$ is cocomplete, and slices of cocomplete categories are themselves cocomplete, so the (filtered) colimit:
                    $$(E_{\infty}, \varphi_{\infty}) := \indlim_{i \in \N} (E_i, \varphi_i)$$
                exists in $\R\-\Vect_{/\R}$. It remains to show that $E_{\infty} \cong E$ and $\varphi_{\infty} \leq \rho$.

                To prove that $E_{\infty} \cong E$, let us suppose to the contrary (for the sake of deriving a contradiction) that one can identify $E_{\infty}$ with a \textit{proper} subspace of $E$. This would imply that there exists some $x \in E \setminus E_{\infty}$. Should we also have that $\varphi_{\infty} \leq \rho$, then we know by lemma \ref{lemma: sublinear_functions_extend_finitely} that one can then simply extend the linear functional $\varphi_{\infty}$ once more, to a linear functional $\varphi_{\infty + 1}: E_{\infty} \oplus \R x \to \R$ such that $\varphi_{\infty + 1} \leq \rho$. The universal property of colimits guarantees, however, that there would be a unique isomorphism $(E_{\infty}, \varphi_{\infty}) \xrightarrow[]{\cong} (E_{\infty} \oplus \R x, \varphi_{})$ in $\R\-\Vect_{/\R}$, which is clearly nonsensical as $x \not = 0$, so we indeed have that $E_{\infty} \cong E$.

                To prove that $\varphi_{\infty} \leq \rho$, simply note that if $\varphi_{\infty} \not \leq \rho$ then there would exist $i \in \N$ and some $x \in E_i$ such that:
                    $$\varphi_{\infty}|_{E_i}(x) := \varphi_i(x) > \rho(x)$$
                which is contradictory to the fact that every $\varphi_i$ is dominated by $\rho$.
            \end{proof}
        \begin{corollary}[Complex Hahn-Banach and norm extension] \label{coro: norm_extensions}
            Now, let $K$ be an arbitrary complete archimedean field again, and let $E$ be a $K$-vector space and let $\varphi_0: E_0 \to K$ be a $K$-linear functional defined on some subspace $E_0 \subseteq E$. Suppose also that $\rho: E \to \R_{\geq 0}$ is a semi-norm. If:
                $$\forall x \in E_0: |\varphi_0(x)| \leq \rho(x)$$
            then there shall exist a linear extension:
                $$\varphi_{\infty}: E \to \R$$
            of $\varphi_0$ which is such that:
                $$\forall x \in E: |\varphi_{\infty}(x)| \leq \rho(x)$$

            In particular, if $E$ is normed and $\rho$ is given by $\rho(x) := \norm{\varphi_0}_{E^*_{\cont}} \norm{x}_E$ then:
                $$\norm{\varphi_0}_{E^*_{\cont}} = \norm{\varphi_{\infty}}_{E^*_{\cont}}$$
        \end{corollary}
            \begin{proof}
                If $K = \R$ then the assertion comes directly from the fact that $\varphi_0(x) \leq |\varphi_0(x)|$ and from theorem \ref{theorem: hahn_banach}.

                If $K = \bbC$ then we can reduce the assertion to the real case by writing $\varphi_0 := \Re(\varphi_0) + i \Im(\varphi_0)$; by linearity, we even have that:
                    $$\Im(\varphi_0) = \Re(i \varphi_0)$$
            \end{proof}

        Another easy corollary of the Hahn-Banach Theorem is that there is a non-trivial (and representable) duality functor:
            $$(-)^* := \Hom_K(-, \R): K\-\Vect \to K\-\Vect^{\op}$$
        with a non-trivial restriction:
            $$(-)^*_{\cont} \cong \Hom_{K, \cont}(-, \R): K\-\Vect_{\norm{-}} \to K\-\Vect_{\norm{-}}^{\op}$$
        down to the subcategory of normed vector spaces and \textit{continuous} linear maps between them, called the \textbf{continuous duality} functor. This restriction maps normed vector spaces $E$ to $E^*_{\cont}$ equipped with the sup-norm. Again, let us remark that even though the functors $(-)^*$ and $(-)^*_{\cont}$ do exist abstractly (as they are just contravariant hom-functors), we do not know before the Hahn-Banach Theorem as to whether or not they might have reasonably large essential images.
        
        The following definition is tautologically equivalent to the definition of contravariant hom-functors, though we state it regardless for the sake of establishing the terminologies. 
        \begin{definition}[Transposition] \label{def: transposition}
            Given a (continuous) $K$-linear map $T: E \to E'$, we call its image $T^*: E'^* \to E^*$ (respectively, $T^*_{\cont}: E'^*_{\cont} \to E^*_{\cont}$) under the (continuous) duality functor the \textbf{(continuous) transpose} of $T$. Explicitly, this is given by:
                $$T^*[\psi](x) := \psi( T(x) )$$
            for all $\psi \in E'^*$ and all $x \in E$ (and likewise for $T^*_{\cont}$).
        \end{definition}

        Let us now investigate the properties of the functor $(-)^*_{\cont}$.
        \begin{convention}
            To avoid notation clutter, the self-composition of $(-)^*_{\cont}$ shall be denoted by $(-)^{**}_{\cont}$.
        \end{convention}
        \begin{lemma}[Continuous duality is involutive] \label{lemma: continuous_duality_involutive}
            The continuous duality functor $(-)^*_{\cont}$ is involutive, i.e. $(-)^{**}_{\cont}$ is the identity functor. 
        \end{lemma}
            \begin{proof}
                Firstly, we shall need to prove that there is an isometry $E \xrightarrow[]{\cong} ( E^*_{\cont} )^*_{\cont}$ for any normed vector space $E$. We claim that this is given by:
                    $$x \mapsto (E^*_{\cont} \xrightarrow[]{\ev_x} \R)$$
                where:
                    $$\ev_x[\varphi] := \varphi(x)$$
                To see that this is an isometry, simple note that the following holds for all $x \in E$ and all $\varphi \in E^*_{\cont}$:
                    $$\norm{\ev_x}_{\sup} := \sup_{\varphi \in E^*_{\cont}, \norm{\varphi}_{E^*_{\cont}} = 1} |\ev_x[\varphi]| = \sup_{\varphi \in E^*_{\cont}, \norm{\varphi}_{E^*_{\cont}} = 1} \abs{\varphi(x)} = \norm{x}_E$$

                Next, let $E, E'$ be normed spaces and denote the canonical isometries between them and their continuous double duals by $\ev$ and $\ev'$ respectively. Let $T: E \to E'$ be a continuous linear map. Then, precisely because $\ev$ and $\ev'$ are isometric, we have the following:
                    $$\norm{ T^{**}_{\cont} }_{\sup} = \norm{ \ev' \circ T \circ \ev^{-1} }_{\sup} := \sup_{y \in E^{**}_{\cont}} \frac{ \norm{ (\ev' \circ T \circ \ev^{-1})(y) }_{\sup} }{\norm{y}_E} = \sup_{y \in E^{**}_{\cont}} \frac{\norm{T(y)}_{E'}}{\norm{y}_E} = \norm{T}_{\sup}$$
                This concludes the proof that there is an isometry:
                    $$\Hom_{K, \cont}(E, E') \xrightarrow[]{\cong} \Hom_{K, \cont}(E^{**}_{\cont}, E'^{**}_{\cont})$$
            \end{proof}
        For the proof of the following proposition, it will be useful to note that if $\Phi: V \to V'$ is any linear map between normed spaces, then:
            $$\forall v \in V: \norm{\Phi}_{V^*} := \sup_{v \in V} \frac{\norm{\Phi(v)}}{\norm{v}_V} \implies \norm{\Phi}_{V^*_{\cont}} \norm{v}_V \geq \abs{\Phi(v)}$$
        \begin{proposition}[Norms of transpositions] \label{prop: transposition_norms}
            Let $T: E \to E'$ be any continuous linear map between normed spaces. Then:
                $$\norm{T}_{\sup} = \norm{T^*_{\cont}}_{\sup}$$
            In fact, the continuous duality functor $(-)_{\cont}^*$ is not only fully faithful, but also isometric on hom-spaces, in the sense that each of the $K$-linear isomorphisms:
                $$\Hom_{K, \cont}(E, E') \xrightarrow[]{\cong} \Hom_{K, \cont}(E'^*_{\cont}, E^*_{\cont})$$
            is moreover an isometry, for all $E, E' \in \Ob( K\-\Vect_{\norm{-}} )$.
        \end{proposition}
            \begin{proof}
                For any $\psi \in E'^*_{\cont}$ and any $x \in E$, we have that:
                    $$\abs{ T^*_{\cont}[\psi](x) } = \abs{\psi(T(x))} \leq \norm{\psi}_{E^*_{\cont}} \norm{T(x)}_{E'} \leq \norm{\psi}_{E'^*_{\cont}} \norm{T}_{\sup} \norm{x}_E$$
                and hence:
                    $$\forall x \in E: \forall \psi \in E'^*_{\cont}: \norm{T}_{\sup} \geq \frac{ \abs{ T^*_{\cont}[\psi](x) } }{\norm{\psi}_{E'^*_{\cont}} \norm{x}_E }$$
                from which one gathers that $\norm{T}_{\sup} \geq \norm{T^*_{\cont}}_{\sup}$. Arguing similarly will yield us $\norm{T^*_{\cont}}_{\sup} \geq \norm{T^{**}_{\cont}}_{\sup}$. But we know from lemma \ref{lemma: continuous_duality_involutive} that $\norm{T^{**}_{\cont}}_{\sup} = \norm{T}_{\sup}$, and hence $\norm{T}_{\sup} \leq \norm{T^*_{\cont}}_{\sup}$ as well, and as such we have shown that $\norm{T}_{\sup} = \norm{T^*_{\cont}}_{\sup}$.
            \end{proof}
        \begin{proposition}[Properties of continuous duals] \label{prop: properties_of_continuous_duals}
            The continuous duality functor $(-)^*_{\cont}$ preserves the following (co)limits in $K\-\Vect_{\norm{-}}$:
            \begin{enumerate}
                \item finite direct sums;
                \item short exact sequences:
                    $$0 \to F \xrightarrow[]{\ker \pi} E \xrightarrow[]{\pi} Q \to 0$$
                where $F \subseteq E$ is a closed subspace, and in fact, we have a linear isometry:
                    $$T: \Ann_{E^*_{\cont}}(F) \xrightarrow[]{\cong} Q^*_{\cont}$$
                determined by:
                    $$\forall x \in E: T[\varphi]( \pi(x) ) := \varphi(\pi(x))$$
                where $\Ann_{E^*}(F) := \{ \varphi \in E^* \mid \varphi(F) = 0 \}$, equipped with the subspace topology; note that there is a small abuse of notations here: the domain of $\varphi$ is not spanned by the vectors $\pi(x) \in Q$, but rather their images under some linear splitting\footnote{Which always exists.} $Q \to E$ of the quotient map $\pi: E \to Q$.
            \end{enumerate}
        \end{proposition}
            \begin{proof}
                \begin{enumerate}
                    \item This is self-evident.
                    \item The definition of the quotient topology ensures that the quotient map $\pi: E \to Q$ is continuous, and so in particular, it preserves convergence of sequences. $\pi$ is also surjective, so any sequence in $Q$ arises as the image of a sequence in $E$. Closedness of $Q^*_{\cont}$ inside $E_{\cont}^*$ via $\pi_{\cont}^*$ then follows from the fact that $\norm{\pi^*_{\cont}}_{\sup} = \norm{\pi}_{\sup}$, as proven in proposition \ref{prop: transposition_norms}, and hence the functor $(-)^*_{\cont}$ maps a short exact sequence:
                        $$0 \to F \xrightarrow[]{\ker \pi} E \xrightarrow[]{\pi} Q \to 0$$
                    in $K\-\Vect_{\norm{-}}$ wherein $F \subseteq E$ is a closed subspace, to the following short exact sequence in $K\-\Vect_{\norm{-}}^{\op}$:
                        $$0 \to Q^*_{\cont} \xrightarrow[]{\pi^*_{\cont}} E^*_{\cont} \xrightarrow[]{\coker \pi^*_{\cont}} F^*_{\cont} \to 0$$
                    wherein $Q^*_{\cont} \subseteq E^*_{\cont}$ is a closed subspace.

                    Lastly, to prove that $T: \Ann_{E^*_{\cont}}(F) \to Q^*_{\cont}$ is an isometry, consider the following for any $\varphi \in \Ann_{E^*_{\cont}}(F)$:
                        $$\norm{T[\varphi]}_{Q^*_{\cont}} := \sup_{x \in E} \frac{ \abs{T[\varphi]( \pi(x) )} }{ \norm{\pi(x)}_Q } = \sup_{x \in E} \frac{ \abs{\varphi(\pi(x))} }{ \norm{\pi(x)}_Q } = \norm{\varphi}_{\Ann_{E^*_{\cont}}(F)}$$
                    wherein the last equality holds because $\varphi(F) = 0$ for all $\varphi \in \Ann_{E^*_{\cont}}(F)$. Linearity of $T$ is self-evident.
                \end{enumerate}
            \end{proof}
        \begin{corollary}
            Let:
                $$0 \to F \xrightarrow[]{\ker \pi} E \xrightarrow[]{\pi} Q \to 0$$
            be a short exact sequence in $K\-\Vect_{\norm{-}}$, wherein $F \subseteq E$ is a closed subspace. Then there is a linear isometry:
                $$F^*_{\cont} \xrightarrow[]{\cong} E^*_{\cont}/\Ann_{E^*_{\cont}}(F)$$
                $$\psi \mapsto \psi|_F$$
        \end{corollary}
        \begin{proposition}[Split monomorphisms of normed spaces: linear closed immersions retract] \label{prop: split_monomorphisms_of_normed_spaces}
            The category $K\-\Vect_{\norm{-}}$ in fact also admits split monomorphisms, which are precisely linear closed immersions.
        \end{proposition}
            \begin{proof}
                If $E$ is any normed space and $F$ is any subspace therein, then:
                    $$E^*_{\cont} \cong F^*_{\cont} \oplus \Ann_{E^*_{\cont}}(F)$$
                by the definition of $\Ann_{E^*_{\cont}}(F)$ as in proposition \ref{prop: properties_of_continuous_duals}. Next, let us consider a short exact sequence in $K\-\Vect_{\norm{-}}$:
                    $$0 \to F \xrightarrow[]{\ker \pi} E \xrightarrow[]{\pi} Q \to 0$$
                wherein $F \subseteq E$ is a closed subspace. In this situation, we now know that there is a linear isometry:
                    $$T: \Ann_{E^*_{\cont}}(F) \xrightarrow[]{\cong} Q^*_{\cont}$$
                and hence there is an isomorphism of normed spaces:
                    $$\id_{F^*_{\cont}} \oplus T: F^*_{\cont} \oplus \Ann_{E^*_{\cont}}(F) \xrightarrow[]{\cong} F^*_{\cont} \oplus Q^*_{\cont}$$
                From this, we see that there is a section:
                    $$s: F^*_{\cont} \to E^*_{\cont}$$
                of $\coker \pi^*_{\cont}: E^*_{\cont} \to F^*_{\cont}$ given by inclusion into the first direct summand. Now, since $(-)^*_{\cont}$ is involutive (see lemma \ref{lemma: continuous_duality_involutive}), we can apply the functor once more to get a linear continuous retract:
                    $$s^*_{\cont}: E \to F$$
                (note the implicit identifications of $E, F$ with the continuous double duals).
            \end{proof}
        \begin{corollary}[Annihilators vs. orthogonal complements] \label{coro: annihilators_vs_orthogonal_complements}
            Let $E$ be a normed space and $F \subseteq E$ be a closed vector subspace thereof. Then, there shall be a linear isometry:
                $$F^{\perp}_{\cont} \cong ( \Ann_{E^*_{\cont}}(F) )^*_{\cont}$$
        \end{corollary}
            \begin{proof}
                The definition of annihlators guarantee us that:
                    $$E^*_{\cont} \cong F^*_{\cont} \oplus \Ann_{E^*_{\cont}}(F)$$
                Since we also know that $(-)^*_{\cont}$ is involutive (see lemma \ref{lemma: continuous_duality_involutive}), meaning that there are linear isometries between $E, F$ and their continuous double duals $E^{**}_{\cont}, F^{**}_{\cont}$, continuously dualising the isomorphism above shall yield\footnote{We are also using the fact that $(-)^*_{\cont}$ preserves finite direct sums (see proposition \ref{prop: properties_of_continuous_duals}).}:
                    $$E \cong F \oplus \Ann_{E^*_{\cont}}(F)^*_{\cont}$$
                and hence there is a linear isometry:
                    $$F^{\perp}_{\cont} \cong ( \Ann_{E^*_{\cont}}(F) )^*_{\cont}$$
                wherein the LHS is the orthogonal complement $F^{\perp}$ of $F$ inside $E$, equipped with the subspace topology inherited from the norm topology on $E$.
            \end{proof}
        \begin{remark}[Orthogonal complements are continuous sections]
            Let $E$ be a normed space and $F \subseteq E$ be a closed vector subspace thereof.
        
            Note that by construction, $\Ann_{E^*_{\cont}}(F)$ is a closed vector subspace of $E^*$ (for any vector subspace $F \subseteq E$ actually, not just closed ones) and fits into the following short exact sequence:
                $$0 \to \Ann_{E^*_{\cont}}(F) \to E^*_{\cont} \to F^*_{\cont} \to 0$$
            where the arrow $E^*_{\cont} \to F^*_{\cont}$ is the image under $(-)^*_{\cont}$ of the inclusion $F \hookrightarrow E$. This, in turn, gives rise to a short exact sequence:
                $$0 \to F \to E \to ( \Ann_{E^*_{\cont}}(F) )^*_{\cont} \to 0$$
            Since $\Ann_{E^*_{\cont}}(F)$ is a closed subspace of $E^*_{\cont}$, its inclusion into $E^*_{\cont}$ admits a retract $E^*_{\cont} \to \Ann_{E^*_{\cont}}(F)$, which gives rise to a section $( \Ann_{E^*_{\cont}}(F) )^*_{\cont} \to E$. The image of this section is nothing but $F^{\perp}_{\cont}$, per the definition of orthogonal complements. As such, we see that corollary \ref{coro: annihilators_vs_orthogonal_complements} really is a consequence of the fact that $K\-\Vect_{\norm{-}}$ admits split monomorphisms, as shown in proposition \ref{prop: split_monomorphisms_of_normed_spaces}.
        \end{remark}
        
        \begin{lemma}[A density criterion] \label{lemma: density_orthogonal_complement_criterion}
            Let $E$ be a normed space. A vector subspace $W \subseteq E$ is dense if and only if $W^{\perp}_{\cont} \cong 0$.
        \end{lemma}
            \begin{proof}
                Suppose firstly that we have a dense vector subspace $W \subseteq E$. This is to say that any point $x \in E$ is the limit of some Cauchy sequence $\{w_n\}_{n \geq 0} \subset W$. Then, for all $\varphi \in \Ann_{E^*}(W)$, we shall have the following as a consequence of $\varphi$ being continuous:
                    $$\varphi(x) = \lim_{n \to +\infty} \varphi(w_n) = 0$$
                But this implies that $\varphi(x) = 0$ for all $x \in E$, i.e. $\varphi = 0$. From this, we infer that:
                    $$\Ann_{E^*}(W) \cong 0$$
                which then implies that:
                    $$W^{\perp}_{\cont} \cong 0$$
                by duality (cf. corollary \ref{coro: annihilators_vs_orthogonal_complements}).

                Conversely, if $W^{\perp}_{\cont} \cong 0$, then $W = E$, and hence $\overline{W} = E$ trivially, which means that $W$ is tautologically dense inside $E$.
            \end{proof}

        Now, let $E$ be a normed space and $E^*$ be its algebraic linear dual (without any topology equipped for now). There is an evident bilinear pairing:
            $$\<-, -\>: E^* \tensor_K E \to K$$
        given by:
            $$\forall (\varphi, x) \in E^* \x E: \<\varphi, x\> := \varphi(x)$$
        \begin{question}
            What is the coarsest possible topology (i.e. as few open subsets as possible) that one can equip $E^*$ with so that its elements are continuous as functions $E \to K$ between normed spaces ?
        \end{question}
        A linear functional $\varphi \in E^*$ is continuous with respect to the norm topologies on $E$ and $K$ if and only if for any sequence $\{x_n\}_{n \geq 0} \subset E$ with limit $x \in E$:
            $$\forall \e > 0: n \gg 0 \implies \abs{ \varphi(x_n) - \varphi(x) } < \e$$
        As $\varphi: E \to K$ is linear, we can write:
            $$\abs{ \varphi(x_n) - \varphi(x) } = \abs{\varphi(x_n - x)}$$
        and hence observe that:
            $$\abs{\varphi(x_n - x)} \leq \norm{\varphi}_{E^*_{\cont}} \norm{x_n - x}_E$$
        for every convergent sequence $\{x_n\}_{n \geq 0} \to x$, since $\norm{\varphi}_{E^*_{\cont}} := \sup_{y \in E} \frac{\abs{\varphi(y)}}{\norm{y}_E}$. Due to the convergence $\{x_n\}_{n \geq 0} \to x$, we also have that:
            $$\forall \e > 0: n \gg 0 \implies \norm{x_n - x}_E < \e$$
        and so to ensure that $\abs{\varphi(x_n) - \varphi(x)} < \e$ for all $\e$ and all $n \gg 0$, we shall need to require that:
            $$\norm{x_n - x}_E < \frac{\e}{\norm{\varphi}_{E^*_{\cont}}}$$
        (the functional $0$ is tautologically continuous, and for all other cases, the RHS is well-defined); note also that $\varphi$ is continuous if and only if it is bounded, so the RHS never vanishes. Since $\e$ is constant, the LHS increases as $\norm{\varphi}_{E^*_{\cont}}$ decreases, and hence $\{\varphi(x_n)\}_{n \geq 0} \to \varphi(x)$ as soon as $x_n$ lies within a relatively \say{large} open ball centered at $x$. The sought-for topology on $E^*$ is thus indeed coarser than the topology induced by the sup-norm $\norm{-}_{E^*_{\cont}}$; in comparison to the former, we may refer to the latter as the \textbf{strong topology}.
        \begin{definition}[Weak topologies] \label{def: weak_topologies}
            Let $E$ be a normed space. A \textbf{weak topology} on $E^*$ is a topology such that any $\varphi \in E^*$, when regarded as a function $\varphi: E \to K$, is continuous with respect to the norm topologies on $E$ and $K$.
        \end{definition}
        \begin{proposition}[Universal property of the weak topology]
            Let $E$ be a normed space. Any weak topology on $E^*$ is actually initial amongst all topologies on $E^*$ such that any $\varphi \in E^*$ is continuous, i.e. any other topology - which is a certain subset of $\calP(E^*)$ - satisfying the condition above contains any weak topology as a subset. Consequently, the weak topology is uniquely defined.
        \end{proposition}
            \begin{proof}
                Let $\calW$ be the set whose elements $w \in \calW$ enumerate all topologies $\tau_w$ on $E^*$ such that any $\varphi \in E^*$ is continuous. Next, pick an arbitrary functional $\varphi \in E^*$ and an open subset $V \subseteq K$. The functional $\varphi$ is continuous with respect to some topology $\tau_w$ if and only if:
                    $$\exists w \in \calW: \varphi^{-1}(V) \in \tau_w$$
                which implies that:
                    $$\forall \varphi \in E^*: \varphi^{-1}(V) \in \weak \iff \varphi^{-1}(V) \in \bigcap_{w \in \calW} \tau_w \iff ( \forall w \in \calW: \varphi^{-1}(V) \in \tau_w )$$
                The rest of the claim then follows.
            \end{proof}
        \begin{convention}
            Let $E$ be a normed space. When $E^*$ is equipped with the weak topology, we shall denote it by $E^*_{\weak}$. 
            
            If a sequence of points $\{x_n\}_{n \geq 0}$, which is not necessarily convergent, is such that $\{\varphi(x_n)\}_{n \geq 0} \to \varphi(x)$ for some $x \in E$ and for all functionals $\varphi \in E^*$, then we will say that the sequence $\{x_n\}_{n \geq 0}$ \textbf{converges weakly} to $x$ and write:
                $$\{x_n\}_{\geq 0} \xrightarrow[]{\weak} x$$
        \end{convention}
        \begin{example}[$L^p$-convergence]
            For more details on $L^p$-spaces, see subsection \ref{subsection: L_p_spaces}.

            Let $(X, \mu)$ be a measure space and let $p, q \in \N_{\geq 1} \cup \{+\infty\}$ be such that $\frac1p + \frac1q = 1$. Then, by theorem \ref{theorem: L_p_space_duality}, we have that:
                $$L^p(X, \mu)^*_{\weak} \cong L^q(X, \mu)$$
            or in formulae, we have that:
                $$\{f_n\}_{n \geq 0} \xrightarrow[]{\weak} f \iff \left( \forall g \in L^q(X, \mu): \left\{ \int_X f_n g d\mu \right\}_{n \geq 0} \to \int_X fg d\mu \right)$$
            for $f_n, f \in L^p(X, \mu)$.
        \end{example}
        \begin{convention}[The weak-$*$ topology]
            There is also something called the \say{weak-$*$ topology}, which is nothing more than the construction of the weak topology on $E^{**}$, regarded as the (algebraic) dual of the normed space $E^*_{\cont}$. We will refrain from using this terminology, as it is somewhat awkward.
        \end{convention}
        Despite being rather coarse, the weak topology is still sufficiently fine and not too pathological.
        \begin{lemma}[Weak topology is Hausdorff] \label{lemma: weak_topology_is_hausdorff}
            For any normed space $E$, the weak topology on $E^*$ is Hausdorff.
        \end{lemma}
            \begin{proof}
                \todo[inline]{Intersection/coarsenings of Hausdorff topologies are not necessarily Hausdorff, since there may not be enough open sets to separate points.}
            \end{proof}
        \begin{theorem}[Weak duality for finite-dimensional normed spaces] \label{theorem: finite_dimensional_weak_duality}
            Let $E$ be a normed space. Then, the weak topology on $E^*$ will coincide with the sup-norm topology (i.e. weak = strong) if and only if $E$ is finite-dimensional.
        \end{theorem}
            \begin{proof}
                
            \end{proof}
        \begin{theorem}[Banach-Alaoglu: bounded sequences converge weakly] \label{theorem: banach_alaoglu}
            Let $E$ be a normed space. Then $E^{**}_{\weak}$ will be compact.
        \end{theorem}
            \begin{proof}
                
            \end{proof}

    \subsection{Hilbert spaces}
        \begin{convention}
            The complex conjugation of some complex number $z := a + ib$ shall be denoted by $z^{\dagger} := a - ib$.
        \end{convention}
    
        \begin{definition}[Hilbert spaces] \label{def: hilbert_spaces}
            Let $\sigma$ be a field automorphism of $K$.
        
            A \textbf{Hilbert space} is a vector space space $H$ equipped with a $\sigma$-braided, positive-definite, and non-degenerate $K$-bilinear form $\<-, -\>_H: H \x H \to K$ (called an \textbf{inner product}), with the braidedness condition meaning that:
                $$\forall x, y \in H: \<x, y\>_H = \sigma( \<y, x\>_H )$$
            The inner product $\<-, -\>$ is said to be \textbf{symmetric} if and only if $\sigma = \id_K$, and \textbf{Hermitian} if and only if $\sigma = (-)^{\dagger}$; note that for complete archimedean local fields, these are the only two possible field automorphisms (since $\Gal(\bbC/\R) \cong \Z/2$). 
        \end{definition}
        \begin{proposition}[Induced norms] \label{prop: induced_norms_on_hilbert_spaces}
            Let $H$ be a Hilbert space. Then, there will be an induced norm $\norm{-}_H$ on $H$ given by:
                $$\forall x \in H: \norm{x}_H := \sqrt{\<x, x\>_H}$$
            When equipped with the topology induced by this norm, Hilbert spaces become normed spaces. 
        \end{proposition}
            \begin{proof}
                Firstly, let us note that $\norm{-}_H$ is a well-defined function $H \to \R_{\geq 0}$ due to the positive-semi-definiteness of $\<-, -\>_H$. Next, the triangle inequality is due to the Cauchy-Schwarz inequality. Finally, to show that $\norm{x}_H = 0$ if and only if $x = 0$, we can simply make use of positive-definiteness.
            \end{proof}
        \begin{convention}
            \textit{A priori}, Hilbert spaces are not complete (at least according to definition \ref{def: hilbert_spaces}; it is common in the literature to require that Hilbert spaces are complete right from the beginning, but we find this to be circular\footnote{From the beginning, Hilbert spaces do not carry any topology, so it does not make any sense to require them to be complete. Only after we have shown that inner products induce norms on Hilbert spaces can we meaningfully require them to be complete.}). However, let us assume from now on that every Hilbert space is complete with respect to its norm topology. In particular, this means that Hilbert spaces are now Banach spaces whose norms come from inner products.
        \end{convention}
        The following is arguably the most ubiquitous and most useful family of examples of Hilbert spaces.  
        \begin{example}[$L^2$-spaces] \label{example: L_2_spaces_as_hilbert_spaces}
            Let $(X, \mu)$ be a measure space. In theorem \ref{theorem: L_p_space_duality}, it is shown that for any $p, q \in \N_{\geq 1} \cup \{+\infty\}$ such that $\frac1p + \frac1q = 1$, one has a linear homeomorphism:
                $$L^q(X, \mu) \xrightarrow[]{\cong} L^p(X, \mu)^*_{\weak}$$
            that is given by:
                $$g \mapsto \int_X (-) g d\mu$$

            Now, observe that when $p = q = 2$, not only does this weak duality holds, but it is in fact a weak \textit{self}-duality of $L^2(X, \mu)$, which allows us to define a bilinear pairing on this Banach space by:
                $$\<f, g\>_{L^2(X, \mu)} := \int_X \abs{ fg } d\mu$$
            using which the linear homeomorphism from before can be now alternatively given by $g \mapsto \<-, g\>$. One then verifies that:
                $$\norm{f}_{L^2(X, \mu)}^2 = \int_X \abs{f}^2 d\mu = \<f, f\>_{L^2(X, \mu)}$$
            to see that the induced norm coincides with the $L^2$-norm.
        \end{example}
        \begin{example}[Finite-dimensional Hilbert spaces]
            Any finite-dimensional inner product space is a Hilbert space. In fact, any finite-dimensional vector space can be upgraded to a Hilbert space in a canonical manner: by letting the inner product be the dot product (but of course, there are inner products that are not the dot product).
        \end{example}

        \begin{theorem}[Riesz's Representability Theorem] \label{theorme: riesz_representation_theorem}
            Let $H$ be a Hilbert space. Then, any continuous linear functional $\varphi \in H^*_{\cont}$ will be representable, in the sense that there exists a unique $y_{\varphi} \in H$ such that:
                $$\varphi = \<-, y_{\varphi}\>_H$$
            In other words, for all $y \in H$, the linear functional $\<-, y\>_H$ is continuous with respect to the topology generated by $\norm{-}_H$, and one has a linear isometry\footnote{\say{$D$} for \say{duality}.}:
                $$D_H: H \xrightarrow[]{\cong} H^*_{\cont}$$
            which is given by $y \mapsto \<-, y\>_H$.
        \end{theorem}
            \begin{proof}
                Let us exploit the fact that there is a linear isometry $\ev: H \xrightarrow[]{\cong} H^{**}_{\cont}$ given by $x \mapsto \ev_x$ with $\ev_x: H^*_{\cont} \to K$ being given by $\ev_x[\varphi] := \varphi(x)$ for all $\varphi \in H^*_{\cont}$ (see lemma \ref{lemma: continuous_duality_involutive}) in order to identify elements of $H$ with those of $H^{**}_{\cont}$ by means of pulling back along this linear isometry. It now suffices to construct a linear isometry:
                    $$P_H: H^{**}_{\cont} \to H^*_{\cont}$$
                We claim that the underlying linear isomorphism can be given by:
                    $$P_H(\ev_y) := \<-, y\>_H$$
                (we will need to verify the continuity of each of the functional $\<-, y\>_H$ too, to be able to conclude that the codomain of $P_H$ is actually $H^*_{\cont}$, not merely all of $H^*$); linearity is easy to see, so let us focus on proving that it is bijective.
                
                Firstly, to prove that $P_H$ is injective:
                    $$\ker P_H \cong \{ y \in H \mid \forall x \in H: \<x, y\>_H = 0 \}$$
                Since the bilinear form $\<-, -\>_H$ is non-degenerate, we can conclude immediately that:
                    $$\ker P_H \cong 0$$
                i.e. that $P_H$ is injective. This also guarantees that any vector $y_{\varphi} \in H$ such that:
                    $$\varphi = \<-, y_{\varphi}\>_H$$
                is necessarily unique: $\ev: H \xrightarrow[]{\cong} H^{**}_{\cont}$ is a linear isometry - so in particular, it is injective - meaning that the composition $P_H \circ \ev$ is also injective.
                
                To see that $P_H$ is surjective, let us note first of all that each $\<-, y\>_H: H \to K$ is continuous by virtue of being bounded (which is per the definition of inner products), and hence $\im P_H \subseteq H^*_{\cont}$. Next, observe that $P_H$ is continuous: to prove this, note firstly that since $H$ is complete by virtue of being a Hilbert space, any Cauchy sequence is convergent\footnote{Hilbert spaces are normed space, hence Hausdorff, so limit points will be unique if they exist.}, and then let $\{y_m\}_{m \geq 0} \to y$ is a (convergent) Cauchy sequence in $H$ and then consider\footnote{We are using the fact that $\{\ev_{y_m}\}_{m \geq 0} \to \ev_y$ if and only if $\{y_m\}_{m \geq 0} \to y$ as the map $\ev: H \to H^{**}_{\cont}$ is an isometry.}\footnote{We leave it to the reader to show that $\norm{ \<x, -\>_H }_{H^*_{\cont}} = \norm{x}_H$ is true for all $x \in H$.}:
                    $$\forall \e > 0: m, n \gg 0 \implies \abs{P_H(\ev_{y_m} - \ev_{y_n})(x)} = \abs{ \< x, y_m - y_n \>_H } \leq \norm{y_m - y_n}_H \norm{ \<x, -\>_H }_{H^*_{\cont}} < \e \norm{x}_H$$
                for all $x \in H$, from which one sees that:
                    $$\forall \e > 0: m, n \gg 0 \implies \norm{ P_H(\ev_{y_m} - \ev_{y_n}) }_{H^*_{\cont}} < \e$$
                which proves that $P_H$ is continuous. We now claim that $\im P_H$ is dense inside $H^*_{\cont}$; this will help us prove surjectivity because $\{\ev_{y_m}\}_{m \geq 0} \to \ev_y$ if and only if $\{y_m\}_{m \geq 0} \to y$ for Cauchy sequences in $H^{**}_{\cont}$ and in $H$ respectively, as the map $\ev: H \to H^{**}_{\cont}$ is an isometry, and hence for any $\varphi \in H^*_{\cont}$, there exists a unique $y_{\varphi} \in H$ such that $\varphi = \<-, y_{\varphi}\>$. To do this, we will be using lemma \ref{lemma: density_orthogonal_complement_criterion}, which tells us that it shall suffice to show that:
                    $$(\im P_H)^{\perp}_{\cont} \cong 0$$
                Let us suppose for the sake of deriving a contradiction that $(\im P_H)^{\perp}_{\cont} \not \cong 0$, i.e.:
                    $$\exists \varphi \in H^*_{\cont} \setminus \{0\}: \varphi(\im P_H) = 0$$
                The construction of $P_H$, however, stipulates that there is a linear isometry:
                    $$H \cong \im P_H$$
                which suggests to us that:
                    $$\varphi(\im P_H) = 0 \iff (\forall x \in H: \varphi(x) = 0) \iff \varphi = 0$$
                But this contradicts the assumption that $\varphi \not = 0$, so it must be the case that:
                    $$(\im P_H)^{\perp}_{\cont} \cong 0$$
                and hence $\im P_H$ is dense inside $H^*_{\cont}$.

                Finally, because we have that:
                    $$\forall y \in H: \norm{ \<-, y\>_H }_{H^*_{\cont}} = \norm{y}_H = \norm{\ev_y}_{H^{**}_{\cont}}$$
                (the last equality is due to $\ev$ being an isometry) there is indeed a linear isometry:
                    $$D_H: H \to H^*_{\cont}$$
                which we now know to be given by $D_H := P_H \circ \ev$. 
            \end{proof}
        \begin{corollary}[Uniformity of Hilbert spaces] \label{coro: hilbert_space_uniformity}
            In Hilbert spaces, sequences converge if and only if they converge weakly. Phrased dually, sequences of continuous functionals on Hilbert spaces converge uniformly if and only if they converge pointwise.
        \end{corollary}
            \begin{proof}
                Strong convergence automatically implies weak convergence, so let us focus on the converse direction.
            
                To that end, pick a weakly convergent sequence $\{\varphi_n\}_{n \geq 0} \xrightarrow[]{\weak} \varphi$ in $H^*$, which means that:
                    $$\forall x \in H: \forall \e > 0: n \gg 0 \implies \abs{\ev_x[\varphi_n - \varphi]} = \abs{\varphi_n(x) - \varphi(x)} < \e$$
                From this, we infer that:
                    $$\norm{ \varphi_n - \varphi }_{H^*_{\cont}} = \sup_{x \in H, \norm{x}_H = 1} \abs{\varphi_n(x) - \varphi(x)} < \e$$
                which shows that there is strong convergence:
                    $$\{\varphi_n\}_{n \geq 0} \to \varphi$$
                Thus, we have shown that weak convergence in $H^*$ implies convergence therein, and since $H^*_{\cont}$ is linearly isometric to $H$ by theorem \ref{theorme: riesz_representation_theorem}, we see thus that the same statement holds for $H$.
            \end{proof}
        \begin{definition}[Gram-Schmidt orthonormalisation]
            
        \end{definition}
        \begin{corollary}[Orthonormal bases] \label{coro: orthonormal_bases_for_hilbert_spaces}
            Every Hilbert space admits a topological orthonormal basis.
        \end{corollary}
            \begin{proof}
                We claim that any topological basis $\{x_i\}_{i \in I}$ for a given Hilbert space $H$ can be refined into an orthonormal one. By theorem \ref{theorme: riesz_representation_theorem}, such a basis induces a dual topological basis $\{x_i^* := \<-, x_i\>_H\}_{i \in I}$ for $H^*_{\cont}$. One can then find a family of vectors $\{e_i\}_{i \in I} \subset H$ such that:
                    $$\forall i \in I: \<e_i, x_j\>_H := \delta_{i, j}$$
                The vectors $e_i$ can be obtained by the Gram-Schmidt procedure, which also guarantees that they are linearly independent from one another. Finally, to show that $\overline{\bigoplus_{i \in I} \bbC e_i} \cong H$ (i.e. that $\{e_i\}_{i \in I}$ indeed \textit{topological} spans $H$), let us note that by the Gram-Schmidt procedure, this is the case if and only if $\{\<-, x_i\>\}_{i \in I}$ is a topological basis for $H^*_{\cont}$. But by corollary \ref{coro: hilbert_space_uniformity}, this is the case if and only if $\{x_i\}_{i \in I}$ is a topological basis for $H$, which is true by assumption.
            \end{proof}

        Let us end the subsection with some examples. 
        \begin{example}[Riesz-Markov-Kakutani representation theorem: measures are functionals]
            
        \end{example}
        \begin{example}[$L^2$-spaces over compact Lie groups and Fourier series; Plancherel's Theorem]
            
        \end{example}