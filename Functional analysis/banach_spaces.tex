\section{Normed spaces}
    Throughout, $K$ shall be a complete archimedean field, so either $\R$ or $\bbC$, and the absolute value on $K$ will be denoted by $\abs{-}$. All vector spaces, unless stated otherwise, shall be $K$-vector spaces.

    \begin{convention}
        To conform to a traditional convention in the theory of integral equations, in calculus of variations, and in probability theory, the arguments of functions whose inputs are linear functionals will be placed between square brackets $[]$ as opposed to the usual parentheses $()$. For example, if $V, W$ are vector spaces and $T: V \to W$ is a linear map then the evaluation of the dual map $T^*: W^* \to V^*$ on functionals $\psi \in W^*$ will be written:
            $$T[\psi^*]$$
        instead of $T(\psi^*)$. This may also help with readability in certain cases. 
    \end{convention}
 
    \subsection{Generalities on topological vector spaces}
        \begin{remark}[Quotient norms] \label{remark: quotient_norms}
            Let $\pi: E \to Q$ be a continuous quotient map of normed spaces. Elements of $\bar{x} \in Q$ (for each of which there is an element $x \in E$, due to $\pi$ being surjective) can be written as cosets:
                $$\bar{x} := x + \ker \pi$$
            for any choice of representative $x \in \pi^{-1}(\bar{x})$. From this and from the fact that $Q$ has the finest possible topology such that $\pi$ is continuous, one sees that:
                $$\norm{\bar{x}}_Q := \inf_{y \in \ker \pi} \norm{x - y}_E = \dist(x, \ker \pi)$$
        \end{remark}
        \begin{lemma}[(Co)limits of normed spaces]
            The category $K\-\Vect_{\norm{-}}$ has the following (co)limits:
            \begin{enumerate}
                \item epimorphisms,
                \item 
            \end{enumerate}
        \end{lemma}
            \begin{proof}
                \begin{enumerate}
                    \item 
                    \item 
                \end{enumerate}
            \end{proof}

    \subsection{The Hahn-Banach Theorem and duality}
        One fundamental problem in the theory of normed spaces is the question of whether the dual space may actually just be trivial. We know this to not be true always (e.g. for any measure space $(X, \mu)$, the spaces $L^p(X, \mu)$ and $L^q(X, \mu)$ are both infinite-dimensional - hence non-trivial - and are dual if $\frac1p + \frac1q = 1$) and in fact, as the Hahn-Banach Theorem will show, the dual space tends to be quite large, provided that the normed is \say{controlled} somehow.

        For convenience, let us begin by introducing the following terminology:
        \begin{definition}[Sublinearity] \label{def: sublinearity}
            A function $\rho: E \to \R$ on a vector space $E$ is said to be \textbf{sublinear} if and only if:
            \begin{itemize}
                \item $\rho$ satisfies the triangle inequality, i.e.:
                    $$\forall x, y \in X: \rho(x + y) \leq \rho(x) + \rho(y)$$
                \item $\rho$ preserves scalar multiplication, i.e.:
                    $$\forall \lambda \in K: \forall x \in E: \rho(\lambda x) = \lambda \rho(x)$$
            \end{itemize}
        \end{definition}
        \begin{example}
            Norms and semi-norms are sublinear. 
        \end{example}
        \begin{definition}[Dominance] \label{def: dominance}
            A real-valued function $\rho: X \to \R$ on a set $X$ is said to \textbf{dominant} another such function $f: X \to \R$ if:
                $$\forall x \in X: f(x) \leq \rho(x)$$
            In such a situation, we shall write:
                $$f \leq \rho$$
        \end{definition}

        The following lemma is very important.
        \begin{lemma}[Sublinear functions extend finitely] \label{lemma: sublinear_functions_extend_finitely}
            Let $K := \R$ and let $E$ be an $\R$-vector space. Let $\rho: E \to \R$ be a sublinear function that dominates an $\R$-linear functional:
                $$\varphi_0: E_0 \to \R$$
            defined on some subspace $E_0 \subseteq E$. Then, there shall exist an extension:
                $$\varphi_1: E_1 \to \R$$
            (i.e. $\varphi_1|_{E_0} = \varphi_0$) that remains linear and dominated by $\rho$, where $E_1 \subseteq E$ is any subspace such that $\dim E_1/E_0 < +\infty$. 
        \end{lemma}
            \begin{proof}
                We can prove this lemma by proving that for any $x_1 \in E \setminus E_0$, there shall exist an extension $\varphi_1: E_0 \oplus \R x_1 \to \R$ that remains linear and dominated by $\rho$. One can then repeat the process $\dim E_1/E_0$ times to get the full assertion. To this end, we shall need to specify how the expressions of the kind below are given:
                    $$\varphi_1( x + \lambda x_1 )$$
                for all $x \in E$ and $\lambda \in \R$. Consider, then, the following, which holds due to definition \ref{def: sublinearity}:
                    $$\rho(x + y) = \rho( x + \lambda x_1 - \lambda x_1 + y ) \leq \rho(x + \lambda x_1) + \rho(-\lambda x_1 + y)$$
                for all $x, y \in E$ and all $\lambda \in \R$. That $\varphi_0$ is linear and that $\varphi_0 \leq \rho$ together tell us that:
                    $$\varphi_0(x) + \varphi_0(y) = \varphi_0(x + y) \leq \rho(x + y)$$
                and hence:
                    $$\varphi_0(x) + \varphi_0(y) \leq \rho(x + \lambda x_1) + \rho(-\lambda x_1 + y)$$
                for all $x, y \in E_0$. Notice, then, that if we were to let:
                    $$\varphi_1(x + \lambda x_1) := \varphi_0(x) + \lambda \alpha_1$$
                for some choice of $\alpha_1 \in \R$ then because we would like:
                    $$\varphi_1(x + \lambda x_1) \leq \rho(x + \lambda x_1)$$
                we will have to choose $\alpha_1$ to be such that:
                    $$\varphi_0(x) + \lambda \alpha_1 \leq \rho(x + \lambda x_1) \iff \lambda \alpha_1 \leq \rho(x + \lambda x_1) - \varphi_0(x)$$
                for all $x \in E$. Since we only need to specify $\alpha_1 := \varphi_1(x_1)$, we can simply set $\lambda := 1$ and then choose:
                    $$\alpha_1 := \inf_{x \in E} \left( \rho(x + x_1) - \varphi_0(x) \right)$$
            \end{proof}
        \begin{remark}[Some comments about the proof]
            One key detail about lemma \ref{lemma: sublinear_functions_extend_finitely} (and hence also theorem \ref{theorem: hahn_banach}, which depends on said lemma) is that it relies crucially on everything being defined over a totally ordered archimedean field - so that theidea that a sublinear function can dominate a functional would even make sense - and on said archimedean field being complete, so that we can consider infima. 
        \end{remark}
        The Hahn-Banach Theorem tells us that $\R$-linear functionals actually extend all the way by a certain limiting procedure. 
        \begin{theorem}[Hahn-Banach: non-triviality of dual spaces] \label{theorem: hahn_banach}
            Let $K := \R$ and let $E$ be an $\R$-vector space. Let $\rho: E \to \R$ be a sublinear function that dominates an $\R$-linear functional:
                $$\varphi_0: E_0 \to \R$$
            defined on some subspace $E_0 \subseteq E$. Then, there shall exist an extension:
                $$\varphi_{\infty}: E \to \R$$
            (i.e. $\varphi_{\infty}|_{E_0} = \varphi_0$) that remains linear and dominated by $\rho$.

            Since $\varphi_{\infty} \in E^*$, this means that the dual space $E^*$ is non-trivial, since it is known that by picking $E_0$ to be finite-dimensional, one can always be guaranteed that there is some non-trivial $\varphi_0 \in E_0^*$ (and hence the extension $\varphi_{\infty}$ is also non-trivial).
        \end{theorem}
            \begin{proof}
                By lemma \ref{lemma: sublinear_functions_extend_finitely}, we know that there is an ascending chain of subspaces of $E$:
                    $$E_0 \subset E_1 \subset E_2 \subset ...$$
                which is possibly non-terminating and is such that $\dim E_{i + 1}/E_i = 1$ for all $i \in \N$, and each term $E_i$ comes equipped with a linear extension $\varphi_i: E_i \to \R$ that is dominated by the given sublinear function $\rho$. It is clear that by construction, we also have that:
                    $$\varphi_{i + 1|_{E_i}} = \varphi_i$$
                and so we have a filtered diagram $\{(E_i, \varphi_i)\}_{i \in \N}$ in the slice category $\R\-\Vect_{/\R}$. The category $\R\-\Vect$ is cocomplete, and slices of cocomplete categories are themselves cocomplete, so the (filtered) colimit:
                    $$(E_{\infty}, \varphi_{\infty}) := \indlim_{i \in \N} (E_i, \varphi_i)$$
                exists in $\R\-\Vect_{/\R}$. It remains to show that $E_{\infty} \cong E$ and $\varphi_{\infty} \leq \rho$.

                To prove that $E_{\infty} \cong E$, let us suppose to the contrary (for the sake of deriving a contradiction) that one can identify $E_{\infty}$ with a \textit{proper} subspace of $E$. This would imply that there exists some $x \in E \setminus E_{\infty}$. Should we also have that $\varphi_{\infty} \leq \rho$, then we know by lemma \ref{lemma: sublinear_functions_extend_finitely} that one can then simply extend the linear functional $\varphi_{\infty}$ once more, to a linear functional $\varphi_{\infty + 1}: E_{\infty} \oplus \R x \to \R$ such that $\varphi_{\infty + 1} \leq \rho$. The universal property of colimits guarantees, however, that there would be a unique isomorphism $(E_{\infty}, \varphi_{\infty}) \xrightarrow[]{\cong} (E_{\infty} \oplus \R x, \varphi_{})$ in $\R\-\Vect_{/\R}$, which is clearly nonsensical as $x \not = 0$, so we indeed have that $E_{\infty} \cong E$.

                To prove that $\varphi_{\infty} \leq \rho$, simply note that if $\varphi_{\infty} \not \leq \rho$ then there would exist $i \in \N$ and some $x \in E_i$ such that:
                    $$\varphi_{\infty}|_{E_i}(x) := \varphi_i(x) > \rho(x)$$
                which is contradictory to the fact that every $\varphi_i$ is dominated by $\rho$.
            \end{proof}
        \begin{corollary}[Complex Hahn-Banach and norm extension] \label{coro: norm_extensions}
            Now, let $K$ be an arbitrary complete archimedean field again, and let $E$ be a $K$-vector space and let $\varphi_0: E_0 \to K$ be a $K$-linear functional defined on some subspace $E_0 \subseteq E$. Suppose also that $\rho: E \to \R_{\geq 0}$ is a semi-norm. If:
                $$\forall x \in E_0: |\varphi_0(x)| \leq \rho(x)$$
            then there shall exist a linear extension:
                $$\varphi_{\infty}: E \to \R$$
            of $\varphi_0$ which is such that:
                $$\forall x \in E: |\varphi_{\infty}(x)| \leq \rho(x)$$

            In particular, if $E$ is normed and $\rho$ is given by $\rho(x) := \norm{\varphi_0}_{E^*_{\cont}} \norm{x}_E$ then:
                $$\norm{\varphi_0}_{E^*_{\cont}} = \norm{\varphi_{\infty}}_{E^*_{\cont}}$$
        \end{corollary}
            \begin{proof}
                If $K = \R$ then the assertion comes directly from the fact that $\varphi_0(x) \leq |\varphi_0(x)|$ and from theorem \ref{theorem: hahn_banach}.

                If $K = \bbC$ then we can reduce the assertion to the real case by writing $\varphi_0 := \Re(\varphi_0) + i \Im(\varphi_0)$; by linearity, we even have that:
                    $$\Im(\varphi_0) = \Re(i \varphi_0)$$
            \end{proof}

        Another easy corollary of the Hahn-Banach Theorem is that there is a non-trivial (and representable) duality functor:
            $$(-)^* := \Hom_K(-, \R): K\-\Vect \to K\-\Vect^{\op}$$
        with a non-trivial restriction:
            $$(-)^*_{\cont} \cong \Hom_{K, \cont}(-, \R): K\-\Vect_{\norm{-}} \to K\-\Vect_{\norm{-}}^{\op}$$
        down to the subcategory of normed vector spaces and \textit{continuous} linear maps between them, called the \textbf{continuous duality} functor. This restriction maps normed vector spaces $E$ to $E^*_{\cont}$ equipped with the sup-norm. Again, let us remark that even though the functors $(-)^*$ and $(-)^*_{\cont}$ do exist abstractly (as they are just contravariant hom-functors), we do not know before the Hahn-Banach Theorem as to whether or not they might have reasonably large essential images.
        
        The following definition is tautologically equivalent to the definition of contravariant hom-functors, though we state it regardless for the sake of establishing the terminologies. 
        \begin{definition}[Transposition] \label{def: transposition}
            Given a (continuous) $K$-linear map $T: E \to E'$, we call its image $T^*: E'^* \to E^*$ (respectively, $T^*_{\cont}: E'^*_{\cont} \to E^*_{\cont}$) under the (continuous) duality functor the \textbf{(continuous) transpose} of $T$. Explicitly, this is given by:
                $$T^*[\psi](x) := \psi( T(x) )$$
            for all $\psi \in E'^*$ and all $x \in E$ (and likewise for $T^*_{\cont}$).
        \end{definition}

        Let us now investigate the properties of the functor $(-)^*_{\cont}$.
        \begin{convention}
            To avoid notation clutter, the self-composition of $(-)^*_{\cont}$ shall be denoted by $(-)^{**}_{\cont}$.
        \end{convention}
        \begin{lemma}[Continuous duality is involutive] \label{lemma: continuous_duality_involutive}
            The continuous duality functor $(-)^*_{\cont}$ is involutive, i.e. $(-)^{**}_{\cont}$ is the identity functor. 
        \end{lemma}
            \begin{proof}
                Firstly, we shall need to prove that there is an isometry $E \xrightarrow[]{\cong} ( E^*_{\cont} )^*_{\cont}$ for any normed vector space $E$. We claim that this is given by:
                    $$x \mapsto (E^*_{\cont} \xrightarrow[]{\ev_x} \R)$$
                where:
                    $$\ev_x[\varphi] := \varphi(x)$$
                To see that this is an isometry, simple note that the following holds for all $x \in E$ and all $\varphi \in E^*_{\cont}$:
                    $$\norm{\ev_x}_{\sup} := \sup_{\varphi \in E^*_{\cont}, \norm{\varphi}_{E^*_{\cont}} = 1} |\ev_x[\varphi]| = \sup_{\varphi \in E^*_{\cont}, \norm{\varphi}_{E^*_{\cont}} = 1} \abs{\varphi(x)} = \norm{x}_E$$

                Next, let $E, E'$ be normed spaces and denote the canonical isometries between them and their continuous double duals by $\ev$ and $\ev'$ respectively. Let $T: E \to E'$ be a continuous linear map. Then, precisely because $\ev$ and $\ev'$ are isometric, we have the following:
                    $$\norm{ T^{**}_{\cont} }_{\sup} = \norm{ \ev' \circ T \circ \ev^{-1} }_{\sup} := \sup_{y \in E^{**}_{\cont}} \frac{ \norm{ (\ev' \circ T \circ \ev^{-1})(y) }_{\sup} }{\norm{y}_E} = \sup_{y \in E^{**}_{\cont}} \frac{\norm{T(y)}_{E'}}{\norm{y}_E} = \norm{T}_{\sup}$$
                This concludes the proof that there is an isometry:
                    $$\Hom_{K, \cont}(E, E') \xrightarrow[]{\cong} \Hom_{K, \cont}(E^{**}_{\cont}, E'^{**}_{\cont})$$
            \end{proof}
        For the proof of the following proposition, it will be useful to note that if $\Phi: V \to V'$ is any linear map between normed spaces, then:
            $$\forall v \in V: \norm{\Phi}_{V^*} := \sup_{v \in V} \frac{\norm{\Phi(v)}}{\norm{v}_V} \implies \norm{\Phi}_{V^*_{\cont}} \norm{v}_V \geq \abs{\Phi(v)}$$
        \begin{proposition}[Norms of transpositions] \label{prop: transposition_norms}
            Let $T: E \to E'$ be any continuous linear map between normed spaces. Then:
                $$\norm{T}_{\sup} = \norm{T^*_{\cont}}_{\sup}$$
            In fact, the continuous duality functor $(-)_{\cont}^*$ is not only fully faithful, but also isometric on hom-spaces, in the sense that each of the $K$-linear isomorphisms:
                $$\Hom_{K, \cont}(E, E') \xrightarrow[]{\cong} \Hom_{K, \cont}(E'^*_{\cont}, E^*_{\cont})$$
            is moreover an isometry, for all $E, E' \in \Ob( K\-\Vect_{\norm{-}} )$.
        \end{proposition}
            \begin{proof}
                For any $\psi \in E'^*_{\cont}$ and any $x \in E$, we have that:
                    $$\abs{ T^*_{\cont}[\psi](x) } = \abs{\psi(T(x))} \leq \norm{\psi}_{E^*_{\cont}} \norm{T(x)}_{E'} \leq \norm{\psi}_{E'^*_{\cont}} \norm{T}_{\sup} \norm{x}_E$$
                and hence:
                    $$\forall x \in E: \forall \psi \in E'^*_{\cont}: \norm{T}_{\sup} \geq \frac{ \abs{ T^*_{\cont}[\psi](x) } }{\norm{\psi}_{E'^*_{\cont}} \norm{x}_E }$$
                from which one gathers that $\norm{T}_{\sup} \geq \norm{T^*_{\cont}}_{\sup}$. Arguing similarly will yield us $\norm{T^*_{\cont}}_{\sup} \geq \norm{T^{**}_{\cont}}_{\sup}$. But we know from lemma \ref{lemma: continuous_duality_involutive} that $\norm{T^{**}_{\cont}}_{\sup} = \norm{T}_{\sup}$, and hence $\norm{T}_{\sup} \leq \norm{T^*_{\cont}}_{\sup}$ as well, and as such we have shown that $\norm{T}_{\sup} = \norm{T^*_{\cont}}_{\sup}$.
            \end{proof}
        \begin{proposition}[Properties of continuous duals] \label{prop: properties_of_continuous_duals}
            The continuous duality functor $(-)^*_{\cont}$ preserves the following (co)limits in $K\-\Vect_{\norm{-}}$:
            \begin{enumerate}
                \item finite direct sums;
                \item short exact sequences:
                    $$0 \to F \xrightarrow[]{\ker \pi} E \xrightarrow[]{\pi} Q \to 0$$
                where $F \subseteq E$ is a closed subspace, and in fact, we have a linear isometry:
                    $$T: \Ann_{E^*_{\cont}}(F) \xrightarrow[]{\cong} Q^*_{\cont}$$
                determined by:
                    $$\forall x \in E: T[\varphi]( \pi(x) ) := \varphi(\pi(x))$$
                where $\Ann_{E^*}(F) := \{ \varphi \in E^* \mid \varphi(F) = 0 \}$, equipped with the subspace topology; note that there is a small abuse of notations here: the domain of $\varphi$ is not spanned by the vectors $\pi(x) \in Q$, but rather their images under some linear splitting\footnote{Which always exists.} $Q \to E$ of the quotient map $\pi: E \to Q$.
            \end{enumerate}
        \end{proposition}
            \begin{proof}
                \begin{enumerate}
                    \item This is self-evident.
                    \item The definition of the quotient topology ensures that the quotient map $\pi: E \to Q$ is continuous, and so in particular, it preserves convergence of sequences. $\pi$ is also surjective, so any sequence in $Q$ arises as the image of a sequence in $E$. Closedness of $Q^*_{\cont}$ inside $E_{\cont}^*$ via $\pi_{\cont}^*$ then follows from the fact that $\norm{\pi^*_{\cont}}_{\sup} = \norm{\pi}_{\sup}$, as proven in proposition \ref{prop: transposition_norms}, and hence the functor $(-)^*_{\cont}$ maps a short exact sequence:
                        $$0 \to F \xrightarrow[]{\ker \pi} E \xrightarrow[]{\pi} Q \to 0$$
                    in $K\-\Vect_{\norm{-}}$ wherein $F \subseteq E$ is a closed subspace, to the following short exact sequence in $K\-\Vect_{\norm{-}}^{\op}$:
                        $$0 \to Q^*_{\cont} \xrightarrow[]{\pi^*_{\cont}} E^*_{\cont} \xrightarrow[]{\coker \pi^*_{\cont}} F^*_{\cont} \to 0$$
                    wherein $Q^*_{\cont} \subseteq E^*_{\cont}$ is a closed subspace.

                    Lastly, to prove that $T: \Ann_{E^*_{\cont}}(F) \to Q^*_{\cont}$ is an isometry, consider the following for any $\varphi \in \Ann_{E^*_{\cont}}(F)$:
                        $$\norm{T[\varphi]}_{Q^*_{\cont}} := \sup_{x \in E} \frac{ \abs{T[\varphi]( \pi(x) )} }{ \norm{\pi(x)}_Q } = \sup_{x \in E} \frac{ \abs{\varphi(\pi(x))} }{ \norm{\pi(x)}_Q } = \norm{\varphi}_{\Ann_{E^*_{\cont}}(F)}$$
                    wherein the last equality holds because $\varphi(F) = 0$ for all $\varphi \in \Ann_{E^*_{\cont}}(F)$. Linearity of $T$ is self-evident.
                \end{enumerate}
            \end{proof}
        \begin{corollary}
            Let:
                $$0 \to F \xrightarrow[]{\ker \pi} E \xrightarrow[]{\pi} Q \to 0$$
            be a short exact sequence in $K\-\Vect_{\norm{-}}$, wherein $F \subseteq E$ is a closed subspace. Then there is a linear isometry:
                $$F^*_{\cont} \xrightarrow[]{\cong} E^*_{\cont}/\Ann_{E^*_{\cont}}(F)$$
                $$\psi \mapsto \psi|_F$$
        \end{corollary}
        \begin{remark}[Split monomorphisms of normed spaces: linear closed immersions retract]
            Via proposition \ref{prop: properties_of_continuous_duals}, we see that the category $K\-\Vect_{\norm{-}}$ in fact also admits split monomorphisms, which are precisely linear closed immersions.

            If $E$ is any normed space and $F$ is any subspace therein, then:
                $$E^*_{\cont} \cong F^*_{\cont} \oplus \Ann_{E^*_{\cont}}(F)$$
            by the definition of $\Ann_{E^*_{\cont}}(F)$ as in proposition \ref{prop: properties_of_continuous_duals}. Next, let us consider a short exact sequence in $K\-\Vect_{\norm{-}}$:
                $$0 \to F \xrightarrow[]{\ker \pi} E \xrightarrow[]{\pi} Q \to 0$$
            wherein $F \subseteq E$ is a closed subspace. In this situation, we now know that there is a linear isometry:
                $$T: \Ann_{E^*_{\cont}}(F) \xrightarrow[]{\cong} Q^*_{\cont}$$
            and hence there is an isomorphism of normed spaces:
                $$\id_{F^*_{\cont}} \oplus T: F^*_{\cont} \oplus \Ann_{E^*_{\cont}}(F) \xrightarrow[]{\cong} F^*_{\cont} \oplus Q^*_{\cont}$$
            From this, we see that there is a section:
                $$s: F^*_{\cont} \to E^*_{\cont}$$
            of $\coker \pi^*_{\cont}: E^*_{\cont} \to F^*_{\cont}$ given by inclusion into the first direct summand. Now, since $(-)^*_{\cont}$ is involutive (see lemma \ref{lemma: continuous_duality_involutive}), we can apply the functor once more to get a linear continuous retract:
                $$s^*_{\cont}: E \to F$$
            (note the implicit identifications of $E, F$ with the continuous double duals).
        \end{remark}

        Now, let $E$ be a normed space and $E^*$ be its algebraic linear dual (without any topology equipped for now). There is an evident bilinear pairing:
            $$\<-, -\>: E^* \tensor_K E \to K$$
        given by:
            $$\forall (\varphi, x) \in E^* \x E: \<\varphi, x\> := \varphi(x)$$
        \begin{question}
            What is the coarsest possible topology (i.e. as few open subsets as possible) that one can equip $E^*$ with so that its elements are continuous as functions $E \to K$ between normed spaces ?
        \end{question}
        A linear functional $\varphi \in E^*$ is continuous with respect to the norm topologies on $E$ and $K$ if and only if for any sequence $\{x_n\}_{n \geq 0} \subset E$ with limit $x \in E$:
            $$\forall \e > 0: n \gg 0 \implies \abs{ \varphi(x_n) - \varphi(x) } < \e$$
        As $\varphi: E \to K$ is linear, we can write:
            $$\abs{ \varphi(x_n) - \varphi(x) } = \abs{\varphi(x_n - x)}$$
        and hence observe that:
            $$\abs{\varphi(x_n - x)} \leq \norm{\varphi}_{E^*_{\cont}} \norm{x_n - x}_E$$
        for every convergent sequence $\{x_n\}_{n \geq 0} \to x$, since $\norm{\varphi}_{E^*_{\cont}} := \sup_{y \in E} \frac{\abs{\varphi(y)}}{\norm{y}_E}$. Due to the convergence $\{x_n\}_{n \geq 0} \to x$, we also have that:
            $$\forall \e > 0: n \gg 0 \implies \norm{x_n - x}_E < \e$$
        and so to ensure that $\abs{\varphi(x_n) - \varphi(x)} < \e$ for all $\e$ and all $n \gg 0$, we shall need to require that:
            $$\norm{x_n - x}_E < \frac{\e}{\norm{\varphi}_{E^*_{\cont}}}$$
        (the functional $0$ is tautologically continuous, and for all other cases, the RHS is well-defined); note also that $\varphi$ is continuous if and only if it is bounded, so the RHS never vanishes. Since $\e$ is constant, the LHS increases as $\norm{\varphi}_{E^*_{\cont}}$ decreases, and hence $\{\varphi(x_n)\}_{n \geq 0} \to \varphi(x)$ as soon as $x_n$ lies within a relatively \say{large} open ball centered at $x$. The sought-for topology on $E^*$ is thus indeed coarser than the topology induced by the sup-norm $\norm{-}_{E^*_{\cont}}$; in comparison to the former, we may refer to the latter as the \textbf{strong topology}.
        \begin{definition}[Weak topologies] \label{def: weak_topologies}
            Let $E$ be a normed space. A \textbf{weak topology} on $E^*$ is a topology such that any $\varphi \in E^*$, when regarded as a function $\varphi: E \to K$, is continuous with respect to the norm topologies on $E$ and $K$.
        \end{definition}
        \begin{proposition}[Universal property of the weak topology]
            Let $E$ be a normed space. Any weak topology on $E^*$ is actually initial amongst all topologies on $E^*$ such that any $\varphi \in E^*$ is continuous, i.e. any other topology - which is a certain subset of $\calP(E^*)$ - satisfying the condition above contains any weak topology as a subset. Consequently, the weak topology is uniquely defined.
        \end{proposition}
            \begin{proof}
                Let $\calW$ be the set whose elements $w \in \calW$ enumerate all topologies $\tau_w$ on $E^*$ such that any $\varphi \in E^*$ is continuous. Next, pick an arbitrary functional $\varphi \in E^*$ and an open subset $V \subseteq K$. The functional $\varphi$ is continuous with respect to some topology $\tau_w$ if and only if:
                    $$\exists w \in \calW: \varphi^{-1}(V) \in \tau_w$$
                which implies that:
                    $$\forall \varphi \in E^*: \varphi^{-1}(V) \in \weak \iff \varphi^{-1}(V) \in \bigcap_{w \in \calW} \tau_w \iff ( \forall w \in \calW: \varphi^{-1}(V) \in \tau_w )$$
                The rest of the claim then follows.
            \end{proof}
        \begin{convention}
            Let $E$ be a normed space. When $E^*$ is equipped with the weak topology, we shall denote it by $E^*_{\weak}$. 
            
            If a sequence of points $\{x_n\}_{n \geq 0}$, which is not necessarily convergent, is such that $\{\varphi(x_n)\}_{n \geq 0} \to \varphi(x)$ for some $x \in E$ and for all functionals $\varphi \in E^*$, then we will say that the sequence $\{x_n\}_{n \geq 0}$ \textbf{converges weakly} to $x$ and write:
                $$\{x_n\}_{\geq 0} \xrightarrow[]{\weak} x$$
        \end{convention}
        \begin{example}[$L^p$-convergence]
            For more details on $L^p$-spaces, see subsection \ref{subsection: L_p_spaces}.

            Let $(X, \mu)$ be a measure space and let $p, q \in \N_{\geq 1} \cup \{+\infty\}$ be such that $\frac1p + \frac1q = 1$. Then, by theorem \ref{theorem: L_p_space_duality}, we have that:
                $$L^p(X, \mu)^*_{\weak} \cong L^q(X, \mu)$$
            or in formulae, we have that:
                $$\{f_n\}_{n \geq 0} \xrightarrow[]{\weak} f \iff \left( \forall g \in L^q(X, \mu): \left\{ \int_X f_n g d\mu \right\}_{n \geq 0} \to \int_X fg d\mu \right)$$
            for $f_n, f \in L^p(X, \mu)$.
        \end{example}
        \begin{convention}[The weak-$*$ topology]
            There is also something called the \say{weak-$*$ topology}, which is nothing more than the construction of the weak topology on $E^{**}$, regarded as the (algebraic) dual of the normed space $E^*_{\cont}$. We will refrain from using this terminology, as it is somewhat awkward.
        \end{convention}
        Despite being rather coarse, the weak topology is still sufficiently fine and not too pathological.
        \begin{lemma}[Weak topology is Hausdorff] \label{lemma: weak_topology_is_hausdorff}
            For any normed space $E$, the weak topology on $E^*$ is Hausdorff.
        \end{lemma}
            \begin{proof}
                \todo[inline]{Intersection/coarsenings of Hausdorff topologies is not necessarily Hausdorff, since there may not be enough open sets to separate points.}
            \end{proof}
        \begin{theorem}[Weak duality for finite-dimensional normed spaces] \label{theorem: finite_dimensional_weak_duality}
            Let $E$ be a normed space. Then, the weak topology on $E^*$ will coincide with the sup-norm topology (i.e. weak = strong) if and only if $E$ is finite-dimensional.
        \end{theorem}
            \begin{proof}
                
            \end{proof}
        \begin{theorem}[Banach-Alaoglu: bounded sequences converge weakly] \label{theorem: banach_alaoglu}
            Let $E$ be a normed space. Then $E^{**}_{\weak}$ will be compact.
        \end{theorem}
            \begin{proof}
                
            \end{proof}

    \subsection{The Baire Category Theorem}

    \subsection{Uniform boundedness}

    \subsection{Open mappings}

    \subsection{The Closed Graph Theorem}