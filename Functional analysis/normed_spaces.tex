\section{Normed spaces}
    Throughout, $K$ shall be a complete archimedean field, so either $\R$ or $\bbC$. All vector spaces, unless stated otherwise, shall be $K$-vector spaces. 
 
    \subsection{Generalities on topological vector spaces}

    \subsection{Duality; the Hahn-Banach Theorem}
        One fundamental problem in the theory of normed spaces is the question of whether the dual space may actually just be trivial. We know this to not be true always (e.g. for any measure space $(X, \mu)$, the spaces $L^p(X, \mu)$ and $L^q(X, \mu)$ are both infinite-dimensional - hence non-trivial - and are dual if $\frac1p + \frac1q = 1$) and in fact, as the Hahn-Banach Theorem will show, the dual space tends to be quite large, provided that the normed is \say{controlled} somehow.

        For convenience, let us begin by introducing the following terminology:
        \begin{definition}[Sublinearity] \label{def: sublinearity}
            A function $\rho: E \to \R$ on a vector space $E$ is said to be \textbf{sublinear} if and only if:
            \begin{itemize}
                \item $\rho$ satisfies the triangle inequality, i.e.:
                    $$\forall x, y \in X: \rho(x + y) \leq \rho(x) + \rho(y)$$
                \item $\rho$ preserves scalar multiplication, i.e.:
                    $$\forall \lambda \in K: \forall x \in E: \rho(\lambda x) = \lambda \rho(x)$$
            \end{itemize}
        \end{definition}
        \begin{example}
            Norms and semi-norms are sublinear. 
        \end{example}
        \begin{definition}[Dominance] \label{def: dominance}
            A function $\rho: X \to E$ on a set $X$ into some normed space $(E, \|-\|)$ is said to \textbf{dominant} another such function $f: X \to E$ if:
                $$\forall x \in X: \|f(x)\| \leq \|\rho(x)\|$$
            In such a situation, we shall write:
                $$f \preccurlyeq \rho$$
        \end{definition}

        The following lemma is very important.
        \begin{lemma}[Sublinear functions extend finitely] \label{lemma: sublinear_functions_extend_finitely}
            Let $K := \R$ and let $E$ be an $\R$-vector space. Let $\rho: E \to \R$ be a sublinear function that dominates an $\R$-linear functional:
                $$\varphi: E_0 \to \R$$
            defined on some subspace $E_0 \subseteq E$. Then, there shall exist an extension:
                $$\tilde{\varphi}: E_1 \to \R$$
            that remains linear and dominated by $\rho$, where $E_1 \subseteq E$ is any subspace such that $\dim E_1/E_0 < +\infty$. 
        \end{lemma}
            \begin{proof}
                We can prove this lemma by proving that for any $x_1 \in E \setminus E_0$, there shall exist an extension $\tilde{\varphi}: E_0 \oplus \R x_1 \to \R$ that remains linear and dominated by $\rho$. One can then repeat the process $\dim E_1/E_0$ times to get the full assertion. To this end, we shall need to specify how the expressions of the kind below are given:
                    $$\tilde{\varphi}( x + \lambda x_1 )$$
                for all $x \in E$ and $\lambda \in \R$. Consider, then, the following:
                    $$\rho(x + y) = \rho( x + \lambda x_1 - \lambda x_1 + y ) \leq \rho(x + \lambda x_1) + \rho(-\lambda x_1 + y)$$
            \end{proof}
        \begin{theorem}[Hahn-Banach]
            
        \end{theorem}
            \begin{proof}
                
            \end{proof}

    \subsection{The Baire Category Theorem}

    \subsection{Uniform boundedness}

    \subsection{Open mappings}

    \subsection{The Closed Graph Theorem}